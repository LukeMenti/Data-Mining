\documentclass[11pt]{article}

    \usepackage[breakable]{tcolorbox}
    \usepackage{parskip} % Stop auto-indenting (to mimic markdown behaviour)
    

    % Basic figure setup, for now with no caption control since it's done
    % automatically by Pandoc (which extracts ![](path) syntax from Markdown).
    \usepackage{graphicx}
    % Maintain compatibility with old templates. Remove in nbconvert 6.0
    \let\Oldincludegraphics\includegraphics
    % Ensure that by default, figures have no caption (until we provide a
    % proper Figure object with a Caption API and a way to capture that
    % in the conversion process - todo).
    \usepackage{caption}
    \DeclareCaptionFormat{nocaption}{}
    \captionsetup{format=nocaption,aboveskip=0pt,belowskip=0pt}

    \usepackage{float}
    \floatplacement{figure}{H} % forces figures to be placed at the correct location
    \usepackage{xcolor} % Allow colors to be defined
    \usepackage{enumerate} % Needed for markdown enumerations to work
    \usepackage{geometry} % Used to adjust the document margins
    \usepackage{amsmath} % Equations
    \usepackage{amssymb} % Equations
    \usepackage{textcomp} % defines textquotesingle
    % Hack from http://tex.stackexchange.com/a/47451/13684:
    \AtBeginDocument{%
        \def\PYZsq{\textquotesingle}% Upright quotes in Pygmentized code
    }
    \usepackage{upquote} % Upright quotes for verbatim code
    \usepackage{eurosym} % defines \euro

    \usepackage{iftex}
    \ifPDFTeX
        \usepackage[T1]{fontenc}
        \IfFileExists{alphabeta.sty}{
              \usepackage{alphabeta}
          }{
              \usepackage[mathletters]{ucs}
              \usepackage[utf8x]{inputenc}
          }
    \else
        \usepackage{fontspec}
        \usepackage{unicode-math}
    \fi

    \usepackage{fancyvrb} % verbatim replacement that allows latex
    \usepackage{grffile} % extends the file name processing of package graphics 
                         % to support a larger range
    \makeatletter % fix for old versions of grffile with XeLaTeX
    \@ifpackagelater{grffile}{2019/11/01}
    {
      % Do nothing on new versions
    }
    {
      \def\Gread@@xetex#1{%
        \IfFileExists{"\Gin@base".bb}%
        {\Gread@eps{\Gin@base.bb}}%
        {\Gread@@xetex@aux#1}%
      }
    }
    \makeatother
    \usepackage[Export]{adjustbox} % Used to constrain images to a maximum size
    \adjustboxset{max size={0.9\linewidth}{0.9\paperheight}}

    % The hyperref package gives us a pdf with properly built
    % internal navigation ('pdf bookmarks' for the table of contents,
    % internal cross-reference links, web links for URLs, etc.)
    \usepackage{hyperref}
    % The default LaTeX title has an obnoxious amount of whitespace. By default,
    % titling removes some of it. It also provides customization options.
    \usepackage{titling}
    \usepackage{longtable} % longtable support required by pandoc >1.10
    \usepackage{booktabs}  % table support for pandoc > 1.12.2
    \usepackage{array}     % table support for pandoc >= 2.11.3
    \usepackage{calc}      % table minipage width calculation for pandoc >= 2.11.1
    \usepackage[inline]{enumitem} % IRkernel/repr support (it uses the enumerate* environment)
    \usepackage[normalem]{ulem} % ulem is needed to support strikethroughs (\sout)
                                % normalem makes italics be italics, not underlines
    \usepackage{mathrsfs}
    

    
    % Colors for the hyperref package
    \definecolor{urlcolor}{rgb}{0,.145,.698}
    \definecolor{linkcolor}{rgb}{.71,0.21,0.01}
    \definecolor{citecolor}{rgb}{.12,.54,.11}

    % ANSI colors
    \definecolor{ansi-black}{HTML}{3E424D}
    \definecolor{ansi-black-intense}{HTML}{282C36}
    \definecolor{ansi-red}{HTML}{E75C58}
    \definecolor{ansi-red-intense}{HTML}{B22B31}
    \definecolor{ansi-green}{HTML}{00A250}
    \definecolor{ansi-green-intense}{HTML}{007427}
    \definecolor{ansi-yellow}{HTML}{DDB62B}
    \definecolor{ansi-yellow-intense}{HTML}{B27D12}
    \definecolor{ansi-blue}{HTML}{208FFB}
    \definecolor{ansi-blue-intense}{HTML}{0065CA}
    \definecolor{ansi-magenta}{HTML}{D160C4}
    \definecolor{ansi-magenta-intense}{HTML}{A03196}
    \definecolor{ansi-cyan}{HTML}{60C6C8}
    \definecolor{ansi-cyan-intense}{HTML}{258F8F}
    \definecolor{ansi-white}{HTML}{C5C1B4}
    \definecolor{ansi-white-intense}{HTML}{A1A6B2}
    \definecolor{ansi-default-inverse-fg}{HTML}{FFFFFF}
    \definecolor{ansi-default-inverse-bg}{HTML}{000000}

    % common color for the border for error outputs.
    \definecolor{outerrorbackground}{HTML}{FFDFDF}

    % commands and environments needed by pandoc snippets
    % extracted from the output of `pandoc -s`
    \providecommand{\tightlist}{%
      \setlength{\itemsep}{0pt}\setlength{\parskip}{0pt}}
    \DefineVerbatimEnvironment{Highlighting}{Verbatim}{commandchars=\\\{\}}
    % Add ',fontsize=\small' for more characters per line
    \newenvironment{Shaded}{}{}
    \newcommand{\KeywordTok}[1]{\textcolor[rgb]{0.00,0.44,0.13}{\textbf{{#1}}}}
    \newcommand{\DataTypeTok}[1]{\textcolor[rgb]{0.56,0.13,0.00}{{#1}}}
    \newcommand{\DecValTok}[1]{\textcolor[rgb]{0.25,0.63,0.44}{{#1}}}
    \newcommand{\BaseNTok}[1]{\textcolor[rgb]{0.25,0.63,0.44}{{#1}}}
    \newcommand{\FloatTok}[1]{\textcolor[rgb]{0.25,0.63,0.44}{{#1}}}
    \newcommand{\CharTok}[1]{\textcolor[rgb]{0.25,0.44,0.63}{{#1}}}
    \newcommand{\StringTok}[1]{\textcolor[rgb]{0.25,0.44,0.63}{{#1}}}
    \newcommand{\CommentTok}[1]{\textcolor[rgb]{0.38,0.63,0.69}{\textit{{#1}}}}
    \newcommand{\OtherTok}[1]{\textcolor[rgb]{0.00,0.44,0.13}{{#1}}}
    \newcommand{\AlertTok}[1]{\textcolor[rgb]{1.00,0.00,0.00}{\textbf{{#1}}}}
    \newcommand{\FunctionTok}[1]{\textcolor[rgb]{0.02,0.16,0.49}{{#1}}}
    \newcommand{\RegionMarkerTok}[1]{{#1}}
    \newcommand{\ErrorTok}[1]{\textcolor[rgb]{1.00,0.00,0.00}{\textbf{{#1}}}}
    \newcommand{\NormalTok}[1]{{#1}}
    
    % Additional commands for more recent versions of Pandoc
    \newcommand{\ConstantTok}[1]{\textcolor[rgb]{0.53,0.00,0.00}{{#1}}}
    \newcommand{\SpecialCharTok}[1]{\textcolor[rgb]{0.25,0.44,0.63}{{#1}}}
    \newcommand{\VerbatimStringTok}[1]{\textcolor[rgb]{0.25,0.44,0.63}{{#1}}}
    \newcommand{\SpecialStringTok}[1]{\textcolor[rgb]{0.73,0.40,0.53}{{#1}}}
    \newcommand{\ImportTok}[1]{{#1}}
    \newcommand{\DocumentationTok}[1]{\textcolor[rgb]{0.73,0.13,0.13}{\textit{{#1}}}}
    \newcommand{\AnnotationTok}[1]{\textcolor[rgb]{0.38,0.63,0.69}{\textbf{\textit{{#1}}}}}
    \newcommand{\CommentVarTok}[1]{\textcolor[rgb]{0.38,0.63,0.69}{\textbf{\textit{{#1}}}}}
    \newcommand{\VariableTok}[1]{\textcolor[rgb]{0.10,0.09,0.49}{{#1}}}
    \newcommand{\ControlFlowTok}[1]{\textcolor[rgb]{0.00,0.44,0.13}{\textbf{{#1}}}}
    \newcommand{\OperatorTok}[1]{\textcolor[rgb]{0.40,0.40,0.40}{{#1}}}
    \newcommand{\BuiltInTok}[1]{{#1}}
    \newcommand{\ExtensionTok}[1]{{#1}}
    \newcommand{\PreprocessorTok}[1]{\textcolor[rgb]{0.74,0.48,0.00}{{#1}}}
    \newcommand{\AttributeTok}[1]{\textcolor[rgb]{0.49,0.56,0.16}{{#1}}}
    \newcommand{\InformationTok}[1]{\textcolor[rgb]{0.38,0.63,0.69}{\textbf{\textit{{#1}}}}}
    \newcommand{\WarningTok}[1]{\textcolor[rgb]{0.38,0.63,0.69}{\textbf{\textit{{#1}}}}}
    
    
    % Define a nice break command that doesn't care if a line doesn't already
    % exist.
    \def\br{\hspace*{\fill} \\* }
    % Math Jax compatibility definitions
    \def\gt{>}
    \def\lt{<}
    \let\Oldtex\TeX
    \let\Oldlatex\LaTeX
    \renewcommand{\TeX}{\textrm{\Oldtex}}
    \renewcommand{\LaTeX}{\textrm{\Oldlatex}}
    % Document parameters
    % Document title
    \title{ESAME}
    
    
    
    
    
% Pygments definitions
\makeatletter
\def\PY@reset{\let\PY@it=\relax \let\PY@bf=\relax%
    \let\PY@ul=\relax \let\PY@tc=\relax%
    \let\PY@bc=\relax \let\PY@ff=\relax}
\def\PY@tok#1{\csname PY@tok@#1\endcsname}
\def\PY@toks#1+{\ifx\relax#1\empty\else%
    \PY@tok{#1}\expandafter\PY@toks\fi}
\def\PY@do#1{\PY@bc{\PY@tc{\PY@ul{%
    \PY@it{\PY@bf{\PY@ff{#1}}}}}}}
\def\PY#1#2{\PY@reset\PY@toks#1+\relax+\PY@do{#2}}

\@namedef{PY@tok@w}{\def\PY@tc##1{\textcolor[rgb]{0.73,0.73,0.73}{##1}}}
\@namedef{PY@tok@c}{\let\PY@it=\textit\def\PY@tc##1{\textcolor[rgb]{0.24,0.48,0.48}{##1}}}
\@namedef{PY@tok@cp}{\def\PY@tc##1{\textcolor[rgb]{0.61,0.40,0.00}{##1}}}
\@namedef{PY@tok@k}{\let\PY@bf=\textbf\def\PY@tc##1{\textcolor[rgb]{0.00,0.50,0.00}{##1}}}
\@namedef{PY@tok@kp}{\def\PY@tc##1{\textcolor[rgb]{0.00,0.50,0.00}{##1}}}
\@namedef{PY@tok@kt}{\def\PY@tc##1{\textcolor[rgb]{0.69,0.00,0.25}{##1}}}
\@namedef{PY@tok@o}{\def\PY@tc##1{\textcolor[rgb]{0.40,0.40,0.40}{##1}}}
\@namedef{PY@tok@ow}{\let\PY@bf=\textbf\def\PY@tc##1{\textcolor[rgb]{0.67,0.13,1.00}{##1}}}
\@namedef{PY@tok@nb}{\def\PY@tc##1{\textcolor[rgb]{0.00,0.50,0.00}{##1}}}
\@namedef{PY@tok@nf}{\def\PY@tc##1{\textcolor[rgb]{0.00,0.00,1.00}{##1}}}
\@namedef{PY@tok@nc}{\let\PY@bf=\textbf\def\PY@tc##1{\textcolor[rgb]{0.00,0.00,1.00}{##1}}}
\@namedef{PY@tok@nn}{\let\PY@bf=\textbf\def\PY@tc##1{\textcolor[rgb]{0.00,0.00,1.00}{##1}}}
\@namedef{PY@tok@ne}{\let\PY@bf=\textbf\def\PY@tc##1{\textcolor[rgb]{0.80,0.25,0.22}{##1}}}
\@namedef{PY@tok@nv}{\def\PY@tc##1{\textcolor[rgb]{0.10,0.09,0.49}{##1}}}
\@namedef{PY@tok@no}{\def\PY@tc##1{\textcolor[rgb]{0.53,0.00,0.00}{##1}}}
\@namedef{PY@tok@nl}{\def\PY@tc##1{\textcolor[rgb]{0.46,0.46,0.00}{##1}}}
\@namedef{PY@tok@ni}{\let\PY@bf=\textbf\def\PY@tc##1{\textcolor[rgb]{0.44,0.44,0.44}{##1}}}
\@namedef{PY@tok@na}{\def\PY@tc##1{\textcolor[rgb]{0.41,0.47,0.13}{##1}}}
\@namedef{PY@tok@nt}{\let\PY@bf=\textbf\def\PY@tc##1{\textcolor[rgb]{0.00,0.50,0.00}{##1}}}
\@namedef{PY@tok@nd}{\def\PY@tc##1{\textcolor[rgb]{0.67,0.13,1.00}{##1}}}
\@namedef{PY@tok@s}{\def\PY@tc##1{\textcolor[rgb]{0.73,0.13,0.13}{##1}}}
\@namedef{PY@tok@sd}{\let\PY@it=\textit\def\PY@tc##1{\textcolor[rgb]{0.73,0.13,0.13}{##1}}}
\@namedef{PY@tok@si}{\let\PY@bf=\textbf\def\PY@tc##1{\textcolor[rgb]{0.64,0.35,0.47}{##1}}}
\@namedef{PY@tok@se}{\let\PY@bf=\textbf\def\PY@tc##1{\textcolor[rgb]{0.67,0.36,0.12}{##1}}}
\@namedef{PY@tok@sr}{\def\PY@tc##1{\textcolor[rgb]{0.64,0.35,0.47}{##1}}}
\@namedef{PY@tok@ss}{\def\PY@tc##1{\textcolor[rgb]{0.10,0.09,0.49}{##1}}}
\@namedef{PY@tok@sx}{\def\PY@tc##1{\textcolor[rgb]{0.00,0.50,0.00}{##1}}}
\@namedef{PY@tok@m}{\def\PY@tc##1{\textcolor[rgb]{0.40,0.40,0.40}{##1}}}
\@namedef{PY@tok@gh}{\let\PY@bf=\textbf\def\PY@tc##1{\textcolor[rgb]{0.00,0.00,0.50}{##1}}}
\@namedef{PY@tok@gu}{\let\PY@bf=\textbf\def\PY@tc##1{\textcolor[rgb]{0.50,0.00,0.50}{##1}}}
\@namedef{PY@tok@gd}{\def\PY@tc##1{\textcolor[rgb]{0.63,0.00,0.00}{##1}}}
\@namedef{PY@tok@gi}{\def\PY@tc##1{\textcolor[rgb]{0.00,0.52,0.00}{##1}}}
\@namedef{PY@tok@gr}{\def\PY@tc##1{\textcolor[rgb]{0.89,0.00,0.00}{##1}}}
\@namedef{PY@tok@ge}{\let\PY@it=\textit}
\@namedef{PY@tok@gs}{\let\PY@bf=\textbf}
\@namedef{PY@tok@gp}{\let\PY@bf=\textbf\def\PY@tc##1{\textcolor[rgb]{0.00,0.00,0.50}{##1}}}
\@namedef{PY@tok@go}{\def\PY@tc##1{\textcolor[rgb]{0.44,0.44,0.44}{##1}}}
\@namedef{PY@tok@gt}{\def\PY@tc##1{\textcolor[rgb]{0.00,0.27,0.87}{##1}}}
\@namedef{PY@tok@err}{\def\PY@bc##1{{\setlength{\fboxsep}{\string -\fboxrule}\fcolorbox[rgb]{1.00,0.00,0.00}{1,1,1}{\strut ##1}}}}
\@namedef{PY@tok@kc}{\let\PY@bf=\textbf\def\PY@tc##1{\textcolor[rgb]{0.00,0.50,0.00}{##1}}}
\@namedef{PY@tok@kd}{\let\PY@bf=\textbf\def\PY@tc##1{\textcolor[rgb]{0.00,0.50,0.00}{##1}}}
\@namedef{PY@tok@kn}{\let\PY@bf=\textbf\def\PY@tc##1{\textcolor[rgb]{0.00,0.50,0.00}{##1}}}
\@namedef{PY@tok@kr}{\let\PY@bf=\textbf\def\PY@tc##1{\textcolor[rgb]{0.00,0.50,0.00}{##1}}}
\@namedef{PY@tok@bp}{\def\PY@tc##1{\textcolor[rgb]{0.00,0.50,0.00}{##1}}}
\@namedef{PY@tok@fm}{\def\PY@tc##1{\textcolor[rgb]{0.00,0.00,1.00}{##1}}}
\@namedef{PY@tok@vc}{\def\PY@tc##1{\textcolor[rgb]{0.10,0.09,0.49}{##1}}}
\@namedef{PY@tok@vg}{\def\PY@tc##1{\textcolor[rgb]{0.10,0.09,0.49}{##1}}}
\@namedef{PY@tok@vi}{\def\PY@tc##1{\textcolor[rgb]{0.10,0.09,0.49}{##1}}}
\@namedef{PY@tok@vm}{\def\PY@tc##1{\textcolor[rgb]{0.10,0.09,0.49}{##1}}}
\@namedef{PY@tok@sa}{\def\PY@tc##1{\textcolor[rgb]{0.73,0.13,0.13}{##1}}}
\@namedef{PY@tok@sb}{\def\PY@tc##1{\textcolor[rgb]{0.73,0.13,0.13}{##1}}}
\@namedef{PY@tok@sc}{\def\PY@tc##1{\textcolor[rgb]{0.73,0.13,0.13}{##1}}}
\@namedef{PY@tok@dl}{\def\PY@tc##1{\textcolor[rgb]{0.73,0.13,0.13}{##1}}}
\@namedef{PY@tok@s2}{\def\PY@tc##1{\textcolor[rgb]{0.73,0.13,0.13}{##1}}}
\@namedef{PY@tok@sh}{\def\PY@tc##1{\textcolor[rgb]{0.73,0.13,0.13}{##1}}}
\@namedef{PY@tok@s1}{\def\PY@tc##1{\textcolor[rgb]{0.73,0.13,0.13}{##1}}}
\@namedef{PY@tok@mb}{\def\PY@tc##1{\textcolor[rgb]{0.40,0.40,0.40}{##1}}}
\@namedef{PY@tok@mf}{\def\PY@tc##1{\textcolor[rgb]{0.40,0.40,0.40}{##1}}}
\@namedef{PY@tok@mh}{\def\PY@tc##1{\textcolor[rgb]{0.40,0.40,0.40}{##1}}}
\@namedef{PY@tok@mi}{\def\PY@tc##1{\textcolor[rgb]{0.40,0.40,0.40}{##1}}}
\@namedef{PY@tok@il}{\def\PY@tc##1{\textcolor[rgb]{0.40,0.40,0.40}{##1}}}
\@namedef{PY@tok@mo}{\def\PY@tc##1{\textcolor[rgb]{0.40,0.40,0.40}{##1}}}
\@namedef{PY@tok@ch}{\let\PY@it=\textit\def\PY@tc##1{\textcolor[rgb]{0.24,0.48,0.48}{##1}}}
\@namedef{PY@tok@cm}{\let\PY@it=\textit\def\PY@tc##1{\textcolor[rgb]{0.24,0.48,0.48}{##1}}}
\@namedef{PY@tok@cpf}{\let\PY@it=\textit\def\PY@tc##1{\textcolor[rgb]{0.24,0.48,0.48}{##1}}}
\@namedef{PY@tok@c1}{\let\PY@it=\textit\def\PY@tc##1{\textcolor[rgb]{0.24,0.48,0.48}{##1}}}
\@namedef{PY@tok@cs}{\let\PY@it=\textit\def\PY@tc##1{\textcolor[rgb]{0.24,0.48,0.48}{##1}}}

\def\PYZbs{\char`\\}
\def\PYZus{\char`\_}
\def\PYZob{\char`\{}
\def\PYZcb{\char`\}}
\def\PYZca{\char`\^}
\def\PYZam{\char`\&}
\def\PYZlt{\char`\<}
\def\PYZgt{\char`\>}
\def\PYZsh{\char`\#}
\def\PYZpc{\char`\%}
\def\PYZdl{\char`\$}
\def\PYZhy{\char`\-}
\def\PYZsq{\char`\'}
\def\PYZdq{\char`\"}
\def\PYZti{\char`\~}
% for compatibility with earlier versions
\def\PYZat{@}
\def\PYZlb{[}
\def\PYZrb{]}
\makeatother


    % For linebreaks inside Verbatim environment from package fancyvrb. 
    \makeatletter
        \newbox\Wrappedcontinuationbox 
        \newbox\Wrappedvisiblespacebox 
        \newcommand*\Wrappedvisiblespace {\textcolor{red}{\textvisiblespace}} 
        \newcommand*\Wrappedcontinuationsymbol {\textcolor{red}{\llap{\tiny$\m@th\hookrightarrow$}}} 
        \newcommand*\Wrappedcontinuationindent {3ex } 
        \newcommand*\Wrappedafterbreak {\kern\Wrappedcontinuationindent\copy\Wrappedcontinuationbox} 
        % Take advantage of the already applied Pygments mark-up to insert 
        % potential linebreaks for TeX processing. 
        %        {, <, #, %, $, ' and ": go to next line. 
        %        _, }, ^, &, >, - and ~: stay at end of broken line. 
        % Use of \textquotesingle for straight quote. 
        \newcommand*\Wrappedbreaksatspecials {% 
            \def\PYGZus{\discretionary{\char`\_}{\Wrappedafterbreak}{\char`\_}}% 
            \def\PYGZob{\discretionary{}{\Wrappedafterbreak\char`\{}{\char`\{}}% 
            \def\PYGZcb{\discretionary{\char`\}}{\Wrappedafterbreak}{\char`\}}}% 
            \def\PYGZca{\discretionary{\char`\^}{\Wrappedafterbreak}{\char`\^}}% 
            \def\PYGZam{\discretionary{\char`\&}{\Wrappedafterbreak}{\char`\&}}% 
            \def\PYGZlt{\discretionary{}{\Wrappedafterbreak\char`\<}{\char`\<}}% 
            \def\PYGZgt{\discretionary{\char`\>}{\Wrappedafterbreak}{\char`\>}}% 
            \def\PYGZsh{\discretionary{}{\Wrappedafterbreak\char`\#}{\char`\#}}% 
            \def\PYGZpc{\discretionary{}{\Wrappedafterbreak\char`\%}{\char`\%}}% 
            \def\PYGZdl{\discretionary{}{\Wrappedafterbreak\char`\$}{\char`\$}}% 
            \def\PYGZhy{\discretionary{\char`\-}{\Wrappedafterbreak}{\char`\-}}% 
            \def\PYGZsq{\discretionary{}{\Wrappedafterbreak\textquotesingle}{\textquotesingle}}% 
            \def\PYGZdq{\discretionary{}{\Wrappedafterbreak\char`\"}{\char`\"}}% 
            \def\PYGZti{\discretionary{\char`\~}{\Wrappedafterbreak}{\char`\~}}% 
        } 
        % Some characters . , ; ? ! / are not pygmentized. 
        % This macro makes them "active" and they will insert potential linebreaks 
        \newcommand*\Wrappedbreaksatpunct {% 
            \lccode`\~`\.\lowercase{\def~}{\discretionary{\hbox{\char`\.}}{\Wrappedafterbreak}{\hbox{\char`\.}}}% 
            \lccode`\~`\,\lowercase{\def~}{\discretionary{\hbox{\char`\,}}{\Wrappedafterbreak}{\hbox{\char`\,}}}% 
            \lccode`\~`\;\lowercase{\def~}{\discretionary{\hbox{\char`\;}}{\Wrappedafterbreak}{\hbox{\char`\;}}}% 
            \lccode`\~`\:\lowercase{\def~}{\discretionary{\hbox{\char`\:}}{\Wrappedafterbreak}{\hbox{\char`\:}}}% 
            \lccode`\~`\?\lowercase{\def~}{\discretionary{\hbox{\char`\?}}{\Wrappedafterbreak}{\hbox{\char`\?}}}% 
            \lccode`\~`\!\lowercase{\def~}{\discretionary{\hbox{\char`\!}}{\Wrappedafterbreak}{\hbox{\char`\!}}}% 
            \lccode`\~`\/\lowercase{\def~}{\discretionary{\hbox{\char`\/}}{\Wrappedafterbreak}{\hbox{\char`\/}}}% 
            \catcode`\.\active
            \catcode`\,\active 
            \catcode`\;\active
            \catcode`\:\active
            \catcode`\?\active
            \catcode`\!\active
            \catcode`\/\active 
            \lccode`\~`\~ 	
        }
    \makeatother

    \let\OriginalVerbatim=\Verbatim
    \makeatletter
    \renewcommand{\Verbatim}[1][1]{%
        %\parskip\z@skip
        \sbox\Wrappedcontinuationbox {\Wrappedcontinuationsymbol}%
        \sbox\Wrappedvisiblespacebox {\FV@SetupFont\Wrappedvisiblespace}%
        \def\FancyVerbFormatLine ##1{\hsize\linewidth
            \vtop{\raggedright\hyphenpenalty\z@\exhyphenpenalty\z@
                \doublehyphendemerits\z@\finalhyphendemerits\z@
                \strut ##1\strut}%
        }%
        % If the linebreak is at a space, the latter will be displayed as visible
        % space at end of first line, and a continuation symbol starts next line.
        % Stretch/shrink are however usually zero for typewriter font.
        \def\FV@Space {%
            \nobreak\hskip\z@ plus\fontdimen3\font minus\fontdimen4\font
            \discretionary{\copy\Wrappedvisiblespacebox}{\Wrappedafterbreak}
            {\kern\fontdimen2\font}%
        }%
        
        % Allow breaks at special characters using \PYG... macros.
        \Wrappedbreaksatspecials
        % Breaks at punctuation characters . , ; ? ! and / need catcode=\active 	
        \OriginalVerbatim[#1,codes*=\Wrappedbreaksatpunct]%
    }
    \makeatother

    % Exact colors from NB
    \definecolor{incolor}{HTML}{303F9F}
    \definecolor{outcolor}{HTML}{D84315}
    \definecolor{cellborder}{HTML}{CFCFCF}
    \definecolor{cellbackground}{HTML}{F7F7F7}
    
    % prompt
    \makeatletter
    \newcommand{\boxspacing}{\kern\kvtcb@left@rule\kern\kvtcb@boxsep}
    \makeatother
    \newcommand{\prompt}[4]{
        {\ttfamily\llap{{\color{#2}[#3]:\hspace{3pt}#4}}\vspace{-\baselineskip}}
    }
    

    
    % Prevent overflowing lines due to hard-to-break entities
    \sloppy 
    % Setup hyperref package
    \hypersetup{
      breaklinks=true,  % so long urls are correctly broken across lines
      colorlinks=true,
      urlcolor=urlcolor,
      linkcolor=linkcolor,
      citecolor=citecolor,
      }
    % Slightly bigger margins than the latex defaults
    
    \geometry{verbose,tmargin=1in,bmargin=1in,lmargin=1in,rmargin=1in}
    
    

\begin{document}
    
    \maketitle
    
    

    
    We see that our dataset is mad of \(n=62\) and \(p=13\). These are its
main features of subset of plot. We have one qualitative variables
``animated'' and the others are continuos.

Let's also plot the histogram of box in order to check normality and the
boxplot, otherwise let's apply a log transformation. It easy to see that
is better consider a log-tranformation. So I will refer to box variable
as log(box) in the following analysis.

    \begin{tcolorbox}[breakable, size=fbox, boxrule=1pt, pad at break*=1mm,colback=cellbackground, colframe=cellborder]
\prompt{In}{incolor}{78}{\boxspacing}
\begin{Verbatim}[commandchars=\\\{\}]
\PY{c+c1}{\PYZsh{}\PYZsh{} load the data}

\PY{n+nf}{load}\PY{p}{(}\PY{l+s}{\PYZdq{}}\PY{l+s}{movie.RData\PYZdq{}}\PY{p}{)}
\PY{n+nf}{ls}\PY{p}{(}\PY{p}{)}
\PY{n+nf}{names}\PY{p}{(}\PY{n}{movie}\PY{p}{)}
\PY{n}{mydata}\PY{o}{\PYZlt{}\PYZhy{}}\PY{n}{movie}
\PY{n+nf}{nrow}\PY{p}{(}\PY{n}{mydata}\PY{p}{)}
\PY{n+nf}{ncol}\PY{p}{(}\PY{n}{mydata}\PY{p}{)}

\PY{n}{mydata} \PY{o}{\PYZlt{}\PYZhy{}} \PY{n}{mydata}\PY{p}{[}\PY{p}{,} \PY{n+nf}{c}\PY{p}{(}\PY{l+s}{\PYZsq{}}\PY{l+s}{box\PYZsq{}}\PY{p}{,} \PY{l+s}{\PYZsq{}}\PY{l+s}{budget\PYZsq{}}\PY{p}{,} \PY{l+s}{\PYZsq{}}\PY{l+s}{animated\PYZsq{}}\PY{p}{,}\PY{l+s}{\PYZdq{}}\PY{l+s}{starpower\PYZdq{}} \PY{p}{,} \PY{l+s}{\PYZsq{}}\PY{l+s}{cmngsoon\PYZsq{}}\PY{p}{)}\PY{p}{]}
\PY{n}{mydata}\PY{p}{[}\PY{l+m}{1}\PY{o}{:}\PY{l+m}{3}\PY{p}{,}\PY{p}{]}

\PY{n+nf}{summary}\PY{p}{(}\PY{n}{mydata}\PY{p}{)}
\PY{c+c1}{\PYZsh{}check NA values}
\PY{n+nf}{sum}\PY{p}{(}\PY{n+nf}{is.na}\PY{p}{(}\PY{n}{mydata}\PY{p}{)}\PY{p}{)}
\PY{c+c1}{\PYZsh{}clean from NA}
\PY{c+c1}{\PYZsh{}mydata \PYZlt{}\PYZhy{} na.omit(mydata)}


\PY{c+c1}{\PYZsh{}to remove infinite}
\PY{c+c1}{\PYZsh{}which(mydata\PYZdl{}felice==\PYZdq{}\PYZhy{}Inf\PYZdq{})}
\PY{c+c1}{\PYZsh{}mydata\PYZlt{}\PYZhy{}mydata[\PYZhy{}129,]}
\PY{c+c1}{\PYZsh{}check that they are factors otherwise make it}

\PY{n}{mydata}\PY{o}{\PYZdl{}}\PY{n}{animated}\PY{o}{\PYZlt{}\PYZhy{}}\PY{n+nf}{as.factor}\PY{p}{(}\PY{n}{mydata}\PY{o}{\PYZdl{}}\PY{n}{animated}\PY{p}{)}
\PY{c+c1}{\PYZsh{}\PYZsh{} SE HAI VARIABILI QUALITATIVE CON DEI LIVELLI PLOTTA LA TABELLA }

\PY{n+nf}{par}\PY{p}{(}\PY{n}{mfrow}\PY{o}{=}\PY{n+nf}{c}\PY{p}{(}\PY{l+m}{1}\PY{p}{,}\PY{l+m}{2}\PY{p}{)}\PY{p}{)}
\PY{n+nf}{options}\PY{p}{(}\PY{n}{repr.plot.width} \PY{o}{=} \PY{l+m}{10}\PY{p}{,} \PY{n}{repr.plot.height} \PY{o}{=} \PY{l+m}{4}\PY{p}{)}
\PY{n+nf}{hist}\PY{p}{(}\PY{n}{mydata}\PY{o}{\PYZdl{}}\PY{n}{box}\PY{p}{,} \PY{n}{prob}\PY{o}{=}\PY{k+kc}{TRUE}\PY{p}{,} \PY{n}{xlab}\PY{o}{=}\PY{l+s}{\PYZsq{}}\PY{l+s}{box\PYZsq{}}\PY{p}{,} \PY{n}{main}\PY{o}{=}\PY{l+s}{\PYZsq{}}\PY{l+s}{Histogram\PYZsq{}}\PY{p}{,}\PY{n}{col}\PY{o}{=}\PY{l+s}{\PYZdq{}}\PY{l+s}{lightblue\PYZdq{}}\PY{p}{)}
\PY{n+nf}{boxplot}\PY{p}{(}\PY{n}{mydata}\PY{o}{\PYZdl{}}\PY{n}{box}\PY{p}{,} \PY{n}{xlab}\PY{o}{=}\PY{l+s}{\PYZsq{}}\PY{l+s}{box\PYZsq{}}\PY{p}{,} \PY{n}{main}\PY{o}{=}\PY{l+s}{\PYZsq{}}\PY{l+s}{Boxplot\PYZsq{}}\PY{p}{,}\PY{n}{col}\PY{o}{=}\PY{l+s}{\PYZdq{}}\PY{l+s}{lightblue\PYZdq{}} \PY{p}{)}
\PY{n+nf}{hist}\PY{p}{(}\PY{n+nf}{log}\PY{p}{(}\PY{n}{mydata}\PY{o}{\PYZdl{}}\PY{n}{box}\PY{p}{)}\PY{p}{,} \PY{n}{prob}\PY{o}{=}\PY{k+kc}{TRUE}\PY{p}{,} \PY{n}{xlab}\PY{o}{=}\PY{l+s}{\PYZsq{}}\PY{l+s}{log(box)\PYZsq{}}\PY{p}{,} \PY{n}{main}\PY{o}{=}\PY{l+s}{\PYZsq{}}\PY{l+s}{Histogram \PYZsq{}}\PY{p}{,}\PY{n}{col}\PY{o}{=}\PY{l+s}{\PYZdq{}}\PY{l+s}{lightblue\PYZdq{}}\PY{p}{)}
\PY{n+nf}{boxplot}\PY{p}{(}\PY{n+nf}{log}\PY{p}{(}\PY{n}{mydata}\PY{o}{\PYZdl{}}\PY{n}{box}\PY{p}{)}\PY{p}{,} \PY{n}{xlab}\PY{o}{=}\PY{l+s}{\PYZsq{}}\PY{l+s}{log(box)\PYZsq{}}\PY{p}{,} \PY{n}{main}\PY{o}{=}\PY{l+s}{\PYZsq{}}\PY{l+s}{Boxplot\PYZsq{}}\PY{p}{,}\PY{n}{col}\PY{o}{=}\PY{l+s}{\PYZdq{}}\PY{l+s}{lightblue\PYZdq{}} \PY{p}{)}

\PY{n}{mydata}\PY{o}{\PYZdl{}}\PY{n}{box}\PY{o}{\PYZlt{}\PYZhy{}}\PY{n+nf}{log}\PY{p}{(}\PY{p}{(}\PY{n}{mydata}\PY{o}{\PYZdl{}}\PY{n}{box}\PY{p}{)}\PY{p}{)}
\end{Verbatim}
\end{tcolorbox}

    \begin{enumerate*}
\item 'model.mydata'
\item 'model.mydata2'
\item 'model.mydata3'
\item 'model.mydata4'
\item 'model.mydata5'
\item 'movie'
\item 'mydata'
\end{enumerate*}


    
    \begin{enumerate*}
\item 'box'
\item 'mprating'
\item 'budget'
\item 'starpower'
\item 'sequel'
\item 'action'
\item 'comedy'
\item 'animated'
\item 'horror'
\item 'addict'
\item 'cmngsoon'
\item 'fandango'
\item 'cntwait'
\end{enumerate*}


    
    62

    
    13

    
    \begin{tabular}{r|lllll}
 box & budget & animated & starpower & cmngsoon\\
\hline
	 191.67085 &  28.0     & FALSE     & 19.83     & 10       \\
	 631.06589 & 150.0     & TRUE      & 32.69     & 59       \\
	  54.01605 &  37.4     & FALSE     & 15.69     & 24       \\
\end{tabular}


    
    
    \begin{Verbatim}[commandchars=\\\{\}]
      box              budget        animated    starpower        cmngsoon     
 Min.   :  5.119   Min.   :  5.00   FALSE:56   Min.   : 0.00   Min.   :  2.00  
 1st Qu.: 69.565   1st Qu.: 30.50   TRUE : 6   1st Qu.:12.16   1st Qu.: 19.25  
 Median :169.309   Median : 37.40              Median :18.07   Median : 36.50  
 Mean   :207.207   Mean   : 53.29              Mean   :18.03   Mean   : 78.21  
 3rd Qu.:266.961   3rd Qu.: 60.00              3rd Qu.:24.09   3rd Qu.: 66.00  
 Max.   :709.505   Max.   :200.00              Max.   :36.76   Max.   :594.00  
    \end{Verbatim}

    
    0

    
    \begin{center}
    \adjustimage{max size={0.9\linewidth}{0.9\paperheight}}{output_1_7.png}
    \end{center}
    { \hspace*{\fill} \\}
    
    \begin{center}
    \adjustimage{max size={0.9\linewidth}{0.9\paperheight}}{output_1_8.png}
    \end{center}
    { \hspace*{\fill} \\}
    
    Let also check the relationship between Y and the covariates X:

\begin{itemize}
\tightlist
\item
  between box and budget we see a kind of polynomial relationship
\item
  between box and starpower not a particular one
\item
  between box and cmngsoon we see a kind of linear relationship even if
  different points are spread around.
\end{itemize}

By the way it will be necessary to investigate more.

    \begin{tcolorbox}[breakable, size=fbox, boxrule=1pt, pad at break*=1mm,colback=cellbackground, colframe=cellborder]
\prompt{In}{incolor}{79}{\boxspacing}
\begin{Verbatim}[commandchars=\\\{\}]
\PY{n+nf}{options}\PY{p}{(}\PY{n}{repr.plot.width} \PY{o}{=} \PY{l+m}{15}\PY{p}{,} \PY{n}{repr.plot.height} \PY{o}{=} \PY{l+m}{10}\PY{p}{)}
\PY{n+nf}{pairs}\PY{p}{(}\PY{n}{mydata}\PY{p}{[}\PY{p}{,}\PY{n+nf}{c}\PY{p}{(}\PY{l+m}{1}\PY{p}{,}\PY{l+m}{2}\PY{p}{,}\PY{l+m}{4}\PY{p}{,}\PY{l+m}{5}\PY{p}{)}\PY{p}{]}\PY{p}{,}\PY{n}{col}\PY{o}{=}\PY{l+s}{\PYZdq{}}\PY{l+s}{blue\PYZdq{}}\PY{p}{,}\PY{n}{pch}\PY{o}{=}\PY{l+m}{19}\PY{p}{)}
\end{Verbatim}
\end{tcolorbox}

    \begin{center}
    \adjustimage{max size={0.9\linewidth}{0.9\paperheight}}{output_3_0.png}
    \end{center}
    { \hspace*{\fill} \\}
    
    Regardign the interections between variables we see from the following
plots that there is some overlapping (in the three scatter plots) so
maybe there will not be interaction but we have to check it. As concern
the boxplots: - in the first one from left we don't have a clear
difference between the two medians suggesting not interaction, the width
is different, the whiskers are different and with animated = TRUE box is
higher and we have an outlier for TRUE.

\begin{itemize}
\item
  For the second one from left we can perform the same consideration
  saying that budget is higher for animated, the medians are different
  suggesting that there could be some interactions.The width is
  different between animated=TRUE and animated=FALSE and for FALSE we
  have outliers.
\item
  For the third one from left we don't have a clear difference between
  the two medians suggesting not interactions, the width is different,
  the whiskers are different and with animated = FALSE cmngsoon is
  higher and we have and outliers for animated= FALSE
\item
  For the fourth one from left we can perform the same consideration
  saying that budget is higher for animated=TRUE, the medians are
  different suggesting that there could be some interactions.The
  wihiskers are different between animated=TRUE and animated=FALSE and
  we don't have outliers.
\end{itemize}

In order to check possible interactions we have to investigate more with
regression.

    \begin{tcolorbox}[breakable, size=fbox, boxrule=1pt, pad at break*=1mm,colback=cellbackground, colframe=cellborder]
\prompt{In}{incolor}{80}{\boxspacing}
\begin{Verbatim}[commandchars=\\\{\}]
\PY{c+c1}{\PYZsh{}\PYZsh{} plot con tutte le variabili in funzione di y per vedere eventuale relazione}

\PY{c+c1}{\PYZsh{}\PYZsh{}plot per verificare eventuali interazioni, nel caso di variabili di tipo YES/NO o livelli o classe insomme}
\PY{n+nf}{par}\PY{p}{(}\PY{n}{mfrow}\PY{o}{=}\PY{n+nf}{c}\PY{p}{(}\PY{l+m}{1}\PY{p}{,}\PY{l+m}{3}\PY{p}{)}\PY{p}{)}
\PY{n+nf}{options}\PY{p}{(}\PY{n}{repr.plot.width} \PY{o}{=} \PY{l+m}{10}\PY{p}{,} \PY{n}{repr.plot.height} \PY{o}{=} \PY{l+m}{5}\PY{p}{)}

\PY{n+nf}{plot}\PY{p}{(}\PY{n}{mydata}\PY{o}{\PYZdl{}}\PY{n}{budget}\PY{p}{,} \PY{n}{mydata}\PY{o}{\PYZdl{}}\PY{n}{box}\PY{p}{,} \PY{n}{cex.lab}\PY{o}{=}\PY{l+m}{1.2}\PY{p}{,} \PY{n}{xlab}\PY{o}{=}\PY{l+s}{\PYZsq{}}\PY{l+s}{budget\PYZsq{}}\PY{p}{,} \PY{n}{pch}\PY{o}{=}\PY{l+m}{19}\PY{p}{,}\PY{n}{ylab}\PY{o}{=}\PY{l+s}{\PYZsq{}}\PY{l+s}{box\PYZsq{}}\PY{p}{,}\PY{n}{col}\PY{o}{=}\PY{n}{mydata}\PY{o}{\PYZdl{}}\PY{n}{animated}\PY{p}{)}
\PY{n+nf}{legend}\PY{p}{(}\PY{l+s}{\PYZsq{}}\PY{l+s}{bottomright\PYZsq{}}\PY{p}{,} \PY{n}{col}\PY{o}{=}\PY{n+nf}{c}\PY{p}{(}\PY{l+m}{1}\PY{p}{,}\PY{l+m}{2}\PY{p}{)}\PY{p}{,} \PY{n}{pch}\PY{o}{=}\PY{n+nf}{c}\PY{p}{(}\PY{l+m}{19}\PY{p}{,}\PY{l+m}{19}\PY{p}{)}\PY{p}{,}
        \PY{n}{legend}\PY{o}{=}\PY{n+nf}{c}\PY{p}{(}\PY{l+s}{\PYZsq{}}\PY{l+s}{animated==false\PYZsq{}}\PY{p}{,}\PY{l+s}{\PYZsq{}}\PY{l+s}{animated==true\PYZsq{}}\PY{p}{)}\PY{p}{)}



\PY{n+nf}{plot}\PY{p}{(}\PY{n}{mydata}\PY{o}{\PYZdl{}}\PY{n}{cmngsoon}\PY{p}{,} \PY{n}{mydata}\PY{o}{\PYZdl{}}\PY{n}{box}\PY{p}{,} \PY{n}{cex.lab}\PY{o}{=}\PY{l+m}{1.2}\PY{p}{,} \PY{n}{xlab}\PY{o}{=}\PY{l+s}{\PYZsq{}}\PY{l+s}{cmngsoon\PYZsq{}}\PY{p}{,} \PY{n}{pch}\PY{o}{=}\PY{l+m}{19}\PY{p}{,}\PY{n}{ylab}\PY{o}{=}\PY{l+s}{\PYZsq{}}\PY{l+s}{box\PYZsq{}}\PY{p}{,}\PY{n}{col}\PY{o}{=}\PY{n}{mydata}\PY{o}{\PYZdl{}}\PY{n}{animated}\PY{p}{)}
\PY{n+nf}{legend}\PY{p}{(}\PY{l+s}{\PYZsq{}}\PY{l+s}{bottomright\PYZsq{}}\PY{p}{,} \PY{n}{col}\PY{o}{=}\PY{n+nf}{c}\PY{p}{(}\PY{l+m}{1}\PY{p}{,}\PY{l+m}{2}\PY{p}{)}\PY{p}{,} \PY{n}{pch}\PY{o}{=}\PY{n+nf}{c}\PY{p}{(}\PY{l+m}{19}\PY{p}{,}\PY{l+m}{19}\PY{p}{)}\PY{p}{,}
        \PY{n}{legend}\PY{o}{=}\PY{n+nf}{c}\PY{p}{(}\PY{l+s}{\PYZsq{}}\PY{l+s}{animated==false\PYZsq{}}\PY{p}{,}\PY{l+s}{\PYZsq{}}\PY{l+s}{animated==true\PYZsq{}}\PY{p}{)}\PY{p}{)}



\PY{n+nf}{plot}\PY{p}{(}\PY{n}{mydata}\PY{o}{\PYZdl{}}\PY{n}{starpower}\PY{p}{,} \PY{n}{mydata}\PY{o}{\PYZdl{}}\PY{n}{box}\PY{p}{,} \PY{n}{cex.lab}\PY{o}{=}\PY{l+m}{1.2}\PY{p}{,} \PY{n}{xlab}\PY{o}{=}\PY{l+s}{\PYZsq{}}\PY{l+s}{starpower\PYZsq{}}\PY{p}{,} \PY{n}{pch}\PY{o}{=}\PY{l+m}{19}\PY{p}{,}\PY{n}{ylab}\PY{o}{=}\PY{l+s}{\PYZsq{}}\PY{l+s}{box\PYZsq{}}\PY{p}{,}\PY{n}{col}\PY{o}{=}\PY{n}{mydata}\PY{o}{\PYZdl{}}\PY{n}{animated}\PY{p}{)}
\PY{n+nf}{legend}\PY{p}{(}\PY{n}{x}\PY{o}{=}\PY{l+m}{10}\PY{p}{,}\PY{n}{y}\PY{o}{=}\PY{l+m}{3}\PY{p}{,} \PY{n}{col}\PY{o}{=}\PY{n+nf}{c}\PY{p}{(}\PY{l+m}{1}\PY{p}{,}\PY{l+m}{2}\PY{p}{)}\PY{p}{,} \PY{n}{pch}\PY{o}{=}\PY{n+nf}{c}\PY{p}{(}\PY{l+m}{19}\PY{p}{,}\PY{l+m}{19}\PY{p}{)}\PY{p}{,}
        \PY{n}{legend}\PY{o}{=}\PY{n+nf}{c}\PY{p}{(}\PY{l+s}{\PYZsq{}}\PY{l+s}{animated==false\PYZsq{}}\PY{p}{,}\PY{l+s}{\PYZsq{}}\PY{l+s}{animated==true\PYZsq{}}\PY{p}{)}\PY{p}{)}




\PY{n+nf}{par}\PY{p}{(}\PY{n}{mfrow}\PY{o}{=}\PY{n+nf}{c}\PY{p}{(}\PY{l+m}{1}\PY{p}{,}\PY{l+m}{4}\PY{p}{)}\PY{p}{)}
\PY{c+c1}{\PYZsh{}\PYZsh{} plot boxplot per eventuali iterazioni (basati pero sempre su quello sopra)}
\PY{n+nf}{boxplot}\PY{p}{(}\PY{n}{mydata}\PY{o}{\PYZdl{}}\PY{n}{box}\PY{o}{\PYZti{}} \PY{n}{mydata}\PY{o}{\PYZdl{}}\PY{n}{animated}\PY{p}{,} \PY{n}{las}\PY{o}{=}\PY{l+m}{2}\PY{p}{,} \PY{n}{cex.axis}\PY{o}{=}\PY{l+m}{0.7}\PY{p}{,}\PY{n}{col}\PY{o}{=}\PY{l+s}{\PYZdq{}}\PY{l+s}{lightblue\PYZdq{}}\PY{p}{,}\PY{n}{xlab}\PY{o}{=}\PY{l+s}{\PYZdq{}}\PY{l+s}{animated\PYZdq{}}\PY{p}{,}
\PY{n}{ylab}\PY{o}{=}\PY{l+s}{\PYZsq{}}\PY{l+s}{box\PYZsq{}}\PY{p}{,} \PY{n}{main}\PY{o}{=}\PY{l+s}{\PYZdq{}}\PY{l+s}{box vs animated\PYZdq{}}\PY{p}{)}


\PY{n+nf}{boxplot}\PY{p}{(}\PY{n}{mydata}\PY{o}{\PYZdl{}}\PY{n}{budget}\PY{o}{\PYZti{}} \PY{n}{mydata}\PY{o}{\PYZdl{}}\PY{n}{animated}\PY{p}{,} \PY{n}{las}\PY{o}{=}\PY{l+m}{2}\PY{p}{,} \PY{n}{cex.axis}\PY{o}{=}\PY{l+m}{0.7}\PY{p}{,}\PY{n}{col}\PY{o}{=}\PY{l+s}{\PYZdq{}}\PY{l+s}{lightblue\PYZdq{}}\PY{p}{,}\PY{n}{xlab}\PY{o}{=}\PY{l+s}{\PYZdq{}}\PY{l+s}{animated\PYZdq{}}\PY{p}{,}
\PY{n}{ylab}\PY{o}{=}\PY{l+s}{\PYZsq{}}\PY{l+s}{budget\PYZsq{}}\PY{p}{,} \PY{n}{main}\PY{o}{=}\PY{l+s}{\PYZdq{}}\PY{l+s}{budget vs animated\PYZdq{}}\PY{p}{)}

\PY{n+nf}{boxplot}\PY{p}{(}\PY{n}{mydata}\PY{o}{\PYZdl{}}\PY{n}{cmngsoon}\PY{o}{\PYZti{}} \PY{n}{mydata}\PY{o}{\PYZdl{}}\PY{n}{animated}\PY{p}{,} \PY{n}{las}\PY{o}{=}\PY{l+m}{2}\PY{p}{,} \PY{n}{cex.axis}\PY{o}{=}\PY{l+m}{0.7}\PY{p}{,}\PY{n}{col}\PY{o}{=}\PY{l+s}{\PYZdq{}}\PY{l+s}{lightblue\PYZdq{}}\PY{p}{,}\PY{n}{xlab}\PY{o}{=}\PY{l+s}{\PYZdq{}}\PY{l+s}{animated\PYZdq{}}\PY{p}{,}
\PY{n}{ylab}\PY{o}{=}\PY{l+s}{\PYZsq{}}\PY{l+s}{cmngsoon\PYZsq{}}\PY{p}{,} \PY{n}{main}\PY{o}{=}\PY{l+s}{\PYZdq{}}\PY{l+s}{cmngsoon vs animated\PYZdq{}}\PY{p}{)}

\PY{n+nf}{boxplot}\PY{p}{(}\PY{n}{mydata}\PY{o}{\PYZdl{}}\PY{n}{starpower}\PY{o}{\PYZti{}} \PY{n}{mydata}\PY{o}{\PYZdl{}}\PY{n}{animated}\PY{p}{,} \PY{n}{las}\PY{o}{=}\PY{l+m}{2}\PY{p}{,} \PY{n}{cex.axis}\PY{o}{=}\PY{l+m}{0.7}\PY{p}{,}\PY{n}{col}\PY{o}{=}\PY{l+s}{\PYZdq{}}\PY{l+s}{lightblue\PYZdq{}}\PY{p}{,}\PY{n}{xlab}\PY{o}{=}\PY{l+s}{\PYZdq{}}\PY{l+s}{animated\PYZdq{}}\PY{p}{,}
\PY{n}{ylab}\PY{o}{=}\PY{l+s}{\PYZsq{}}\PY{l+s}{starpower\PYZsq{}}\PY{p}{,} \PY{n}{main}\PY{o}{=}\PY{l+s}{\PYZdq{}}\PY{l+s}{starpower vs animated\PYZdq{}}\PY{p}{)}


\PY{c+c1}{\PYZsh{}par(mfrow=c(1,2))}
\PY{c+c1}{\PYZsh{}\PYZsh{} plot moasic plot}
\PY{c+c1}{\PYZsh{}mosaicplot(table(mydata\PYZdl{}prefer,mydata\PYZdl{}stories), las=2, cex.axis=0.7,col=\PYZdq{}lightblue\PYZdq{},}
\PY{c+c1}{\PYZsh{}xlab=\PYZsq{}\PYZsq{}, main=\PYZsq{}prefer vs stories \PYZsq{})}
\PY{c+c1}{\PYZsh{}mosaicplot(table( mydata\PYZdl{}aircon, mydata\PYZdl{}prefer), las=2, cex.axis=0.7,col=\PYZdq{}lightblue\PYZdq{},}
\PY{c+c1}{\PYZsh{}xlab=\PYZsq{}\PYZsq{}, main=\PYZsq{}aircon vs prefer \PYZsq{})}
\PY{c+c1}{\PYZsh{}mosaicplot(table(mydata\PYZdl{}x1, mydata\PYZdl{}x1\PYZhy{}livello), main=\PYZsq{}x1 vs x1\PYZhy{}livello\PYZsq{},col=\PYZdq{}lightblue\PYZdq{})}
\PY{c+c1}{\PYZsh{}mosaicplot(table(mydata\PYZdl{}x1, mydata\PYZdl{}x2\PYZhy{}livello), main=\PYZsq{}x1 vs x2\PYZhy{}livello\PYZsq{},col=\PYZdq{}lightblue\PYZdq{})}
\PY{c+c1}{\PYZsh{}mosaicplot(table(mydata\PYZdl{}x1\PYZhy{}livello, mydata\PYZdl{}x2\PYZhy{}livello), main=\PYZsq{}x1\PYZhy{}livello vs x2\PYZhy{}livello\PYZsq{},col=\PYZdq{}lightblue\PYZdq{})}
\end{Verbatim}
\end{tcolorbox}

    \begin{center}
    \adjustimage{max size={0.9\linewidth}{0.9\paperheight}}{output_5_0.png}
    \end{center}
    { \hspace*{\fill} \\}
    
    \begin{center}
    \adjustimage{max size={0.9\linewidth}{0.9\paperheight}}{output_5_1.png}
    \end{center}
    { \hspace*{\fill} \\}
    
    \hypertarget{multiple-linear-regression}{%
\subsubsection{Multiple Linear
Regression}\label{multiple-linear-regression}}

After this preliminary analysis we can apply a linear regression. Let's
start with a model with all variables and interactions and then perform
model selection base ond \(P-value\).

So I started with a model including also the interactions between the
covariates then the final model I have obtained: \$box=
+4.421+0.008\emph{budget -2.344}animated* \$\$ I\_\{animated=TRUE\} \$
\(+ 0.002*cmngsoon + 0.052*animated*cmngsoon\)

is the following output. In the table below the 95\% CI for coefficients
is reported.

    \begin{tcolorbox}[breakable, size=fbox, boxrule=1pt, pad at break*=1mm,colback=cellbackground, colframe=cellborder]
\prompt{In}{incolor}{81}{\boxspacing}
\begin{Verbatim}[commandchars=\\\{\}]
\PY{n}{model.mydata} \PY{o}{\PYZlt{}\PYZhy{}} \PY{n+nf}{lm}\PY{p}{(}\PY{n}{box}\PY{o}{\PYZti{}}\PY{n}{budget}\PY{o}{*}\PY{n}{cmngsoon} \PY{o}{+} \PY{n}{budget}\PY{o}{*}\PY{n}{starpower} \PY{o}{+} \PY{n}{budget}\PY{o}{*}\PY{n}{animated}\PY{o}{+} \PY{n}{starpower}\PY{o}{*}\PY{n}{cmngsoon} \PY{o}{+}\PY{n}{animated}\PY{o}{*}\PY{n}{cmngsoon}\PY{o}{+}\PY{n}{starpower}\PY{o}{*}\PY{n}{animated} \PY{p}{,} \PY{n}{data}\PY{o}{=}\PY{n}{mydata}\PY{p}{)}

\PY{c+c1}{\PYZsh{}summary(model.mydata)}
\PY{c+c1}{\PYZsh{}\PYZsh{} rimuovere interazione non significativa (P\PYZhy{}value più alto), ad esempio quella tra x1 e x4. }
\PY{n}{model.mydata2} \PY{o}{\PYZlt{}\PYZhy{}} \PY{n+nf}{update}\PY{p}{(}\PY{n}{model.mydata}\PY{p}{,} \PY{n}{.}\PY{o}{\PYZti{}}\PY{n}{.}\PY{o}{\PYZhy{}}\PY{n}{budget}\PY{o}{:}\PY{n}{animated}\PY{o}{\PYZhy{}}\PY{n}{budget}\PY{o}{:}\PY{n}{cmngsoon}\PY{o}{\PYZhy{}}\PY{n}{budget}\PY{o}{:}\PY{n}{starpower}\PY{o}{\PYZhy{}}\PY{n}{cmngsoon}\PY{o}{:}\PY{n}{starpower}\PY{p}{)}
\PY{c+c1}{\PYZsh{}summary(model.mydata2)}
\PY{c+c1}{\PYZsh{}summary(model.mydata2)}
\PY{n}{model.mydata3} \PY{o}{\PYZlt{}\PYZhy{}} \PY{n+nf}{update}\PY{p}{(}\PY{n}{model.mydata2}\PY{p}{,} \PY{n}{.}\PY{o}{\PYZti{}}\PY{n}{.}\PY{o}{\PYZhy{}}\PY{n}{animated}\PY{o}{:}\PY{n}{starpower}\PY{p}{)}

\PY{c+c1}{\PYZsh{}summary(model.mydata3)}


\PY{c+c1}{\PYZsh{}model.mydata4 \PYZlt{}\PYZhy{} update(model.mydata3, .\PYZti{}.\PYZhy{}stories:prefer)}


\PY{n}{model.mydata41} \PY{o}{\PYZlt{}\PYZhy{}} \PY{n+nf}{update}\PY{p}{(}\PY{n}{model.mydata3}\PY{p}{,} \PY{n}{.}\PY{o}{\PYZti{}}\PY{n}{.}\PY{o}{\PYZhy{}}\PY{n}{starpower}\PY{p}{)}

\PY{n+nf}{summary}\PY{p}{(}\PY{n}{model.mydata41}\PY{p}{)}


\PY{n+nf}{confint}\PY{p}{(}\PY{n}{model.mydata41}\PY{p}{)}
\end{Verbatim}
\end{tcolorbox}

    
    \begin{Verbatim}[commandchars=\\\{\}]

Call:
lm(formula = box \textasciitilde{} budget + cmngsoon + animated + cmngsoon:animated, 
    data = mydata)

Residuals:
     Min       1Q   Median       3Q      Max 
-1.75479 -0.53717  0.07359  0.60779  1.21553 

Coefficients:
                        Estimate Std. Error t value Pr(>|t|)    
(Intercept)            4.4211114  0.1652971  26.746  < 2e-16 ***
budget                 0.0080583  0.0025631   3.144 0.002646 ** 
cmngsoon               0.0019240  0.0007752   2.482 0.016031 *  
animatedTRUE          -2.3436926  0.6569115  -3.568 0.000738 ***
cmngsoon:animatedTRUE  0.0519782  0.0157352   3.303 0.001654 ** 
---
Signif. codes:  0 ‘***’ 0.001 ‘**’ 0.01 ‘*’ 0.05 ‘.’ 0.1 ‘ ’ 1

Residual standard error: 0.7437 on 57 degrees of freedom
Multiple R-squared:  0.4188,	Adjusted R-squared:  0.378 
F-statistic: 10.27 on 4 and 57 DF,  p-value: 2.487e-06

    \end{Verbatim}

    
    \begin{tabular}{r|ll}
  & 2.5 \% & 97.5 \%\\
\hline
	(Intercept) &  4.0901096969 &  4.752113005 \\
	budget &  0.0029257614 &  0.013190791 \\
	cmngsoon &  0.0003718007 &  0.003476288 \\
	animatedTRUE & -3.6591351480 & -1.028250123 \\
	cmngsoon:animatedTRUE &  0.0204690696 &  0.083487393 \\
\end{tabular}


    
    I tried also with polynomial terms for cmngsoon and budget but they are
not significant as shown in the following output ( I also tried removing
as quadratic term for budget keeping it for cmngsoon and viceversa but
nothing changes).

    \begin{tcolorbox}[breakable, size=fbox, boxrule=1pt, pad at break*=1mm,colback=cellbackground, colframe=cellborder]
\prompt{In}{incolor}{82}{\boxspacing}
\begin{Verbatim}[commandchars=\\\{\}]
\PY{n}{model.mydata5} \PY{o}{\PYZlt{}\PYZhy{}} \PY{n+nf}{update}\PY{p}{(}\PY{n}{model.mydata41}\PY{p}{,} \PY{n}{.}\PY{o}{\PYZti{}}\PY{n}{.}\PY{o}{+}\PY{n+nf}{I}\PY{p}{(}\PY{n}{budget}\PY{o}{\PYZca{}}\PY{l+m}{2}\PY{p}{)}\PY{o}{+}\PY{n+nf}{I}\PY{p}{(}\PY{n}{cmngsoon}\PY{o}{\PYZca{}}\PY{l+m}{2}\PY{p}{)}\PY{p}{)}

\PY{n+nf}{summary}\PY{p}{(}\PY{n}{model.mydata5}\PY{p}{)}
\end{Verbatim}
\end{tcolorbox}

    
    \begin{Verbatim}[commandchars=\\\{\}]

Call:
lm(formula = box \textasciitilde{} budget + cmngsoon + animated + I(budget\^{}2) + 
    I(cmngsoon\^{}2) + cmngsoon:animated, data = mydata)

Residuals:
     Min       1Q   Median       3Q      Max 
-1.68843 -0.57156  0.07404  0.61563  1.22495 

Coefficients:
                        Estimate Std. Error t value Pr(>|t|)    
(Intercept)            4.500e+00  2.898e-01  15.527  < 2e-16 ***
budget                 7.299e-03  8.592e-03   0.849 0.399317    
cmngsoon               9.640e-05  2.897e-03   0.033 0.973571    
animatedTRUE          -2.384e+00  6.781e-01  -3.515 0.000888 ***
I(budget\^{}2)            6.146e-06  4.642e-05   0.132 0.895146    
I(cmngsoon\^{}2)          3.641e-06  5.534e-06   0.658 0.513263    
cmngsoon:animatedTRUE  5.242e-02  1.632e-02   3.213 0.002198 ** 
---
Signif. codes:  0 ‘***’ 0.001 ‘**’ 0.01 ‘*’ 0.05 ‘.’ 0.1 ‘ ’ 1

Residual standard error: 0.7541 on 55 degrees of freedom
Multiple R-squared:  0.4234,	Adjusted R-squared:  0.3605 
F-statistic: 6.731 on 6 and 55 DF,  p-value: 2.203e-05

    \end{Verbatim}

    
    From anova we see that we keep model without all the terms and
interaction (the initial one).

    \begin{tcolorbox}[breakable, size=fbox, boxrule=1pt, pad at break*=1mm,colback=cellbackground, colframe=cellborder]
\prompt{In}{incolor}{83}{\boxspacing}
\begin{Verbatim}[commandchars=\\\{\}]
\PY{n+nf}{anova}\PY{p}{(}\PY{n}{model.mydata}\PY{p}{,}\PY{n}{model.mydata41}\PY{p}{)}
\end{Verbatim}
\end{tcolorbox}

    \begin{tabular}{r|llllll}
 Res.Df & RSS & Df & Sum of Sq & F & Pr(>F)\\
\hline
	 51        & 29.82083  & NA        &        NA &        NA &        NA\\
	 57        & 31.52650  & -6        & -1.705674 & 0.4861779 & 0.8155907\\
\end{tabular}


    
    Now we can judge also our model considering the residuals. The graph of
residuals indicates that the model does not have an pretty good fit. In
fact, the first graph (scatter plot of the residuals) doesn't show a
non- deterministic pattern . In addition, the mean of the residuals does
not appear to be 0 and the variance of the residuals does not appear to
be constant, as it should be based on the assumptions that the
regression model places on the ε errors. Furthermore,to complete the
analysis of the residuals, outliers appear to be present as shown from
Cook's distance\textgreater1. So let's remove it and see if something
change.

    \begin{tcolorbox}[breakable, size=fbox, boxrule=1pt, pad at break*=1mm,colback=cellbackground, colframe=cellborder]
\prompt{In}{incolor}{84}{\boxspacing}
\begin{Verbatim}[commandchars=\\\{\}]
\PY{c+c1}{\PYZsh{} subdivide the window into 4 parts, 2 rows and 2 columns , residul plot}
\PY{n+nf}{par}\PY{p}{(}\PY{n}{mfrow}\PY{o}{=}\PY{n+nf}{c}\PY{p}{(}\PY{l+m}{2}\PY{p}{,}\PY{l+m}{2}\PY{p}{)}\PY{p}{)}
\PY{n+nf}{options}\PY{p}{(}\PY{n}{repr.plot.width} \PY{o}{=} \PY{l+m}{15}\PY{p}{,} \PY{n}{repr.plot.height} \PY{o}{=} \PY{l+m}{7}\PY{p}{)}
\PY{n+nf}{plot}\PY{p}{(}\PY{n}{model.mydata41}\PY{p}{,}\PY{n}{col}\PY{o}{=}\PY{l+s}{\PYZdq{}}\PY{l+s}{blue\PYZdq{}}\PY{p}{)}
\PY{c+c1}{\PYZsh{}\PYZsh{}\PYZsh{} plot the Cook\PYZsq{}s distance}
\PY{n+nf}{par}\PY{p}{(}\PY{n}{mfrow}\PY{o}{=}\PY{n+nf}{c}\PY{p}{(}\PY{l+m}{1}\PY{p}{,}\PY{l+m}{2}\PY{p}{)}\PY{p}{)}
\PY{n+nf}{plot}\PY{p}{(}\PY{n}{model.mydata41}\PY{p}{,} \PY{l+m}{4}\PY{p}{,}\PY{n}{col}\PY{o}{=}\PY{l+s}{\PYZdq{}}\PY{l+s}{blue\PYZdq{}}\PY{p}{)}
\PY{n+nf}{plot}\PY{p}{(}\PY{n}{model.mydata41}\PY{p}{,} \PY{l+m}{5}\PY{p}{,}\PY{n}{col}\PY{o}{=}\PY{l+s}{\PYZdq{}}\PY{l+s}{blue\PYZdq{}}\PY{p}{)}
\PY{n}{mydata}\PY{o}{\PYZlt{}\PYZhy{}}\PY{n}{mydata}\PY{p}{[}\PY{l+m}{\PYZhy{}12}\PY{p}{,}\PY{p}{]}
\end{Verbatim}
\end{tcolorbox}

    \begin{center}
    \adjustimage{max size={0.9\linewidth}{0.9\paperheight}}{output_13_0.png}
    \end{center}
    { \hspace*{\fill} \\}
    
    \begin{center}
    \adjustimage{max size={0.9\linewidth}{0.9\paperheight}}{output_13_1.png}
    \end{center}
    { \hspace*{\fill} \\}
    
    Removing the outliers leads the animated to lose significance in the
model (also its interactions), so let's try another fit removing it and
the interactions

    \begin{tcolorbox}[breakable, size=fbox, boxrule=1pt, pad at break*=1mm,colback=cellbackground, colframe=cellborder]
\prompt{In}{incolor}{85}{\boxspacing}
\begin{Verbatim}[commandchars=\\\{\}]
\PY{n}{model.mydata4} \PY{o}{\PYZlt{}\PYZhy{}} \PY{n+nf}{lm}\PY{p}{(}\PY{n}{box} \PY{o}{\PYZti{}} \PY{n}{budget}\PY{o}{+} \PY{n}{cmngsoon} \PY{p}{,} \PY{n}{data}\PY{o}{=}\PY{n}{mydata}\PY{p}{)}
\PY{n+nf}{summary}\PY{p}{(}\PY{n}{model.mydata4}\PY{p}{)}
\PY{n+nf}{confint}\PY{p}{(}\PY{n}{model.mydata4}\PY{p}{)}
\end{Verbatim}
\end{tcolorbox}

    
    \begin{Verbatim}[commandchars=\\\{\}]

Call:
lm(formula = box \textasciitilde{} budget + cmngsoon, data = mydata)

Residuals:
    Min      1Q  Median      3Q     Max 
-1.7335 -0.4642  0.1049  0.5941  1.2165 

Coefficients:
             Estimate Std. Error t value Pr(>|t|)    
(Intercept) 4.3772836  0.1535636  28.505  < 2e-16 ***
budget      0.0092566  0.0021263   4.353 5.52e-05 ***
cmngsoon    0.0018777  0.0007334   2.560   0.0131 *  
---
Signif. codes:  0 ‘***’ 0.001 ‘**’ 0.01 ‘*’ 0.05 ‘.’ 0.1 ‘ ’ 1

Residual standard error: 0.7093 on 58 degrees of freedom
Multiple R-squared:  0.3204,	Adjusted R-squared:  0.297 
F-statistic: 13.67 on 2 and 58 DF,  p-value: 1.364e-05

    \end{Verbatim}

    
    \begin{tabular}{r|ll}
  & 2.5 \% & 97.5 \%\\
\hline
	(Intercept) & 4.0698927765 & 4.684674482 \\
	budget & 0.0050003890 & 0.013512714 \\
	cmngsoon & 0.0004096702 & 0.003345711 \\
\end{tabular}


    
    We obtain a model \(box=4.377+ 0.009* budget +0.002* cmngsoon\). We see
that the residuals in this case are goos and doesn't not present a
deterministic path. In addition, the mean of the residuals does appear
to be almost 0 and the variance of the residuals does appear to be
constant, as it should be based on the assumptions that the regression
model places on the ε errors. Furthermore, the normality of the
residuals is satisfied as highlighted in the second graph: the empirical
quantiles, in fact, does not deviate from the theoretical quantiles of a
standard normal (except for one of the tails).

    \begin{tcolorbox}[breakable, size=fbox, boxrule=1pt, pad at break*=1mm,colback=cellbackground, colframe=cellborder]
\prompt{In}{incolor}{86}{\boxspacing}
\begin{Verbatim}[commandchars=\\\{\}]
\PY{c+c1}{\PYZsh{} subdivide the window into 4 parts, 2 rows and 2 columns , residul plot}
\PY{n+nf}{par}\PY{p}{(}\PY{n}{mfrow}\PY{o}{=}\PY{n+nf}{c}\PY{p}{(}\PY{l+m}{2}\PY{p}{,}\PY{l+m}{2}\PY{p}{)}\PY{p}{)}
\PY{n+nf}{options}\PY{p}{(}\PY{n}{repr.plot.width} \PY{o}{=} \PY{l+m}{15}\PY{p}{,} \PY{n}{repr.plot.height} \PY{o}{=} \PY{l+m}{7}\PY{p}{)}
\PY{n+nf}{plot}\PY{p}{(}\PY{n}{model.mydata4}\PY{p}{,}\PY{n}{col}\PY{o}{=}\PY{l+s}{\PYZdq{}}\PY{l+s}{blue\PYZdq{}}\PY{p}{)}
\PY{c+c1}{\PYZsh{}\PYZsh{}\PYZsh{} plot the Cook\PYZsq{}s distance}
\PY{n+nf}{par}\PY{p}{(}\PY{n}{mfrow}\PY{o}{=}\PY{n+nf}{c}\PY{p}{(}\PY{l+m}{1}\PY{p}{,}\PY{l+m}{2}\PY{p}{)}\PY{p}{)}
\PY{n+nf}{plot}\PY{p}{(}\PY{n}{model.mydata4}\PY{p}{,} \PY{l+m}{4}\PY{p}{,}\PY{n}{col}\PY{o}{=}\PY{l+s}{\PYZdq{}}\PY{l+s}{blue\PYZdq{}}\PY{p}{)}
\PY{n+nf}{plot}\PY{p}{(}\PY{n}{model.mydata4}\PY{p}{,} \PY{l+m}{5}\PY{p}{,}\PY{n}{col}\PY{o}{=}\PY{l+s}{\PYZdq{}}\PY{l+s}{blue\PYZdq{}}\PY{p}{)}
\end{Verbatim}
\end{tcolorbox}

    \begin{center}
    \adjustimage{max size={0.9\linewidth}{0.9\paperheight}}{output_17_0.png}
    \end{center}
    { \hspace*{\fill} \\}
    
    \begin{center}
    \adjustimage{max size={0.9\linewidth}{0.9\paperheight}}{output_17_1.png}
    \end{center}
    { \hspace*{\fill} \\}
    
    Let's compare the model obtained with the outlier with the one without
the outlier using anova ( on the same dataset without the outlier). We
see the model without animated (due to elimination of outlier value) is
preferable. So our final model is :
\(box=4.377+ 0.009* budget +0.003* cmngsoon\)

    \begin{tcolorbox}[breakable, size=fbox, boxrule=1pt, pad at break*=1mm,colback=cellbackground, colframe=cellborder]
\prompt{In}{incolor}{89}{\boxspacing}
\begin{Verbatim}[commandchars=\\\{\}]
\PY{n}{model.mydata41} \PY{o}{\PYZlt{}\PYZhy{}} \PY{n+nf}{lm}\PY{p}{(}\PY{n}{box} \PY{o}{\PYZti{}} \PY{n}{budget} \PY{o}{+} \PY{n}{cmngsoon} \PY{o}{+} \PY{n}{animated} \PY{o}{+} \PY{n}{cmngsoon}\PY{o}{:}\PY{n}{animated}\PY{p}{,}\PY{n}{data}\PY{o}{=}\PY{n}{mydata}\PY{p}{)}
\PY{n+nf}{anova}\PY{p}{(}\PY{n}{model.mydata41}\PY{p}{,}\PY{n}{model.mydata4}\PY{p}{)}
\end{Verbatim}
\end{tcolorbox}

    \begin{tabular}{r|llllll}
 Res.Df & RSS & Df & Sum of Sq & F & Pr(>F)\\
\hline
	 56         & 28.79292   & NA         &         NA &        NA  &        NA \\
	 58         & 29.18307   & -2         & -0.3901517 & 0.3794074  & 0.6860122 \\
\end{tabular}


    
    \hypertarget{gam}{%
\subsection{Gam}\label{gam}}

let's also try with smooth splines to see if we obtaine better model. In
order to find the degrees of freedom I have used the cross validation (9
for budget 4 for cmngsoon). I report just the gam output for cmngsoon
since it is the most interested( for budget splines is not useful based
on p-value). From the output,based on \(P-value\) we see that we don't
need smoothing splines (even if we are in a borderline for cmngsoon
since we have \(p-value\)=0.055 (greater than 0.05 by the way) ). In the
following plot we there are the predictions based on this model.

    \begin{tcolorbox}[breakable, size=fbox, boxrule=1pt, pad at break*=1mm,colback=cellbackground, colframe=cellborder]
\prompt{In}{incolor}{96}{\boxspacing}
\begin{Verbatim}[commandchars=\\\{\}]
\PY{n+nf}{set.seed}\PY{p}{(}\PY{l+m}{111}\PY{p}{)}
\PY{n+nf}{library}\PY{p}{(}\PY{n}{gam}\PY{p}{)}
\PY{c+c1}{\PYZsh{}\PYZsh{} CONSIDERA LE VARIABILI NON LINEARI}
\PY{n}{m.gam} \PY{o}{\PYZlt{}\PYZhy{}} \PY{n+nf}{gam}\PY{p}{(}\PY{n}{box} \PY{o}{\PYZti{}} \PY{n}{budget} \PY{o}{+} \PY{n+nf}{s}\PY{p}{(}\PY{n}{cmngsoon}\PY{p}{,}\PY{l+m}{4}\PY{p}{)}\PY{p}{,}
\PY{c+c1}{\PYZsh{}+s(x5, degree of freedoms find before)*x6 (se hai interazione)}
\PY{c+c1}{\PYZsh{}altre variabili del best model,}
\PY{n}{data}\PY{o}{=}\PY{n}{mydata}\PY{p}{)}
\PY{n+nf}{summary}\PY{p}{(}\PY{n}{m.gam}\PY{p}{)}
\end{Verbatim}
\end{tcolorbox}

    \begin{Verbatim}[commandchars=\\\{\}]
Warning message in model.matrix.default(mt, mf, contrasts):
“non-list contrasts argument ignored”
    \end{Verbatim}

    
    \begin{Verbatim}[commandchars=\\\{\}]

Call: gam(formula = box \textasciitilde{} budget + s(cmngsoon, 4), data = mydata)
Deviance Residuals:
     Min       1Q   Median       3Q      Max 
-1.53341 -0.46280  0.07629  0.50166  1.09710 

(Dispersion Parameter for gaussian family taken to be 0.4628)

    Null Deviance: 42.9432 on 60 degrees of freedom
Residual Deviance: 25.453 on 55 degrees of freedom
AIC: 133.794 

Number of Local Scoring Iterations: 2 

Anova for Parametric Effects
               Df  Sum Sq Mean Sq F value    Pr(>F)    
budget          1  9.0506  9.0506 19.5569 4.652e-05 ***
s(cmngsoon, 4)  1  3.3608  3.3608  7.2622  0.009322 ** 
Residuals      55 25.4530  0.4628                      
---
Signif. codes:  0 ‘***’ 0.001 ‘**’ 0.01 ‘*’ 0.05 ‘.’ 0.1 ‘ ’ 1

Anova for Nonparametric Effects
               Npar Df Npar F   Pr(F)  
(Intercept)                            
budget                                 
s(cmngsoon, 4)       3 2.6867 0.05532 .
---
Signif. codes:  0 ‘***’ 0.001 ‘**’ 0.01 ‘*’ 0.05 ‘.’ 0.1 ‘ ’ 1
    \end{Verbatim}

    
    \hypertarget{conclusion-point-1}{%
\section{Conclusion Point 1}\label{conclusion-point-1}}

We have that: - boxs depend positevly from budget - boxs depend
positevly from cmngsoon

The final model for the subset is:
\(box=4.377+ 0.009* budget +0.003* cmngsoon\).

    \hypertarget{ridge}{%
\subsubsection{RIDGE}\label{ridge}}

Let's consider all the dataset. In the following we plot the
informations about covariates and first three rows of the dataset.

    \begin{tcolorbox}[breakable, size=fbox, boxrule=1pt, pad at break*=1mm,colback=cellbackground, colframe=cellborder]
\prompt{In}{incolor}{121}{\boxspacing}
\begin{Verbatim}[commandchars=\\\{\}]
\PY{c+c1}{\PYZsh{}\PYZsh{} load the data}

\PY{n+nf}{load}\PY{p}{(}\PY{l+s}{\PYZdq{}}\PY{l+s}{movie.RData\PYZdq{}}\PY{p}{)}
\PY{n+nf}{ls}\PY{p}{(}\PY{p}{)}
\PY{n+nf}{names}\PY{p}{(}\PY{n}{movie}\PY{p}{)}
\PY{n}{mydata}\PY{o}{\PYZlt{}\PYZhy{}}\PY{n}{movie}
\PY{n+nf}{nrow}\PY{p}{(}\PY{n}{mydata}\PY{p}{)}
\PY{n+nf}{ncol}\PY{p}{(}\PY{n}{mydata}\PY{p}{)}




\PY{n+nf}{summary}\PY{p}{(}\PY{n}{mydata}\PY{p}{)}
\PY{c+c1}{\PYZsh{}check NA values}
\PY{n+nf}{sum}\PY{p}{(}\PY{n+nf}{is.na}\PY{p}{(}\PY{n}{mydata}\PY{p}{)}\PY{p}{)}
\PY{c+c1}{\PYZsh{}clean from NA}
\PY{n}{mydata} \PY{o}{\PYZlt{}\PYZhy{}} \PY{n+nf}{na.omit}\PY{p}{(}\PY{n}{mydata}\PY{p}{)}

\PY{c+c1}{\PYZsh{}mydata\PYZdl{}box\PYZlt{}\PYZhy{}log((mydata\PYZdl{}box))}
\PY{c+c1}{\PYZsh{}to remove infinite}

\PY{n}{mydata}\PY{o}{\PYZlt{}\PYZhy{}}\PY{n}{mydata}\PY{p}{[}\PY{l+m}{\PYZhy{}12}\PY{p}{,}\PY{p}{]}
\PY{n}{mydata}\PY{p}{[}\PY{l+m}{1}\PY{o}{:}\PY{l+m}{3}\PY{p}{,}\PY{p}{]}
\PY{n}{mydata}\PY{o}{\PYZdl{}}\PY{n}{animated}\PY{o}{\PYZlt{}\PYZhy{}}\PY{n+nf}{as.factor}\PY{p}{(}\PY{n}{mydata}\PY{o}{\PYZdl{}}\PY{n}{animated}\PY{p}{)}
\PY{n}{mydata}\PY{o}{\PYZdl{}}\PY{n}{sequel}\PY{o}{\PYZlt{}\PYZhy{}}\PY{n+nf}{as.factor}\PY{p}{(}\PY{n}{mydata}\PY{o}{\PYZdl{}}\PY{n}{sequel}\PY{p}{)}
\PY{n}{mydata}\PY{o}{\PYZdl{}}\PY{n}{action}\PY{o}{\PYZlt{}\PYZhy{}}\PY{n+nf}{as.factor}\PY{p}{(}\PY{n}{mydata}\PY{o}{\PYZdl{}}\PY{n}{action}\PY{p}{)}
\PY{n}{mydata}\PY{o}{\PYZdl{}}\PY{n}{horror}\PY{o}{\PYZlt{}\PYZhy{}}\PY{n+nf}{as.factor}\PY{p}{(}\PY{n}{mydata}\PY{o}{\PYZdl{}}\PY{n}{horror}\PY{p}{)}
\PY{n}{mydata}\PY{o}{\PYZdl{}}\PY{n}{comedy}\PY{o}{\PYZlt{}\PYZhy{}}\PY{n+nf}{as.factor}\PY{p}{(}\PY{n}{mydata}\PY{o}{\PYZdl{}}\PY{n}{comedy}\PY{p}{)}
\end{Verbatim}
\end{tcolorbox}

    \begin{enumerate*}
\item 'X'
\item 'best.lambda'
\item 'cv.lasso'
\item 'cv.ridge'
\item 'id.zero'
\item 'm.gam'
\item 'm.glm'
\item 'm.glm.cv'
\item 'm.lasso'
\item 'm.lasso.min'
\item 'm.ridge'
\item 'm.ridge.min'
\item 'model.mydata'
\item 'model.mydata2'
\item 'model.mydata3'
\item 'model.mydata4'
\item 'model.mydata41'
\item 'model.mydata5'
\item 'movie'
\item 'mse'
\item 'mydata'
\item 'nonzero'
\item 'prediction.lasso'
\item 'prediction.ridge'
\item 'x1.cv'
\item 'x2.cv'
\item 'y'
\end{enumerate*}


    
    \begin{enumerate*}
\item 'box'
\item 'mprating'
\item 'budget'
\item 'starpower'
\item 'sequel'
\item 'action'
\item 'comedy'
\item 'animated'
\item 'horror'
\item 'addict'
\item 'cmngsoon'
\item 'fandango'
\item 'cntwait'
\end{enumerate*}


    
    62

    
    13

    
    
    \begin{Verbatim}[commandchars=\\\{\}]
      box          mprating     budget         starpower       sequel  
 Min.   :  5.119   1: 2     Min.   :  5.00   Min.   : 0.00   FALSE:53  
 1st Qu.: 69.565   2:15     1st Qu.: 30.50   1st Qu.:12.16   TRUE : 9  
 Median :169.309   3:28     Median : 37.40   Median :18.07             
 Mean   :207.207   4:17     Mean   : 53.29   Mean   :18.03             
 3rd Qu.:266.961            3rd Qu.: 60.00   3rd Qu.:24.09             
 Max.   :709.505            Max.   :200.00   Max.   :36.76             
   action     comedy    animated    horror       addict         cmngsoon     
 FALSE:48   FALSE:42   FALSE:56   FALSE:56   Min.   :  568   Min.   :  2.00  
 TRUE :14   TRUE :20   TRUE : 6   TRUE : 6   1st Qu.: 1671   1st Qu.: 19.25  
                                             Median : 3480   Median : 36.50  
                                             Mean   : 5934   Mean   : 78.21  
                                             3rd Qu.: 7836   3rd Qu.: 66.00  
                                             Max.   :45866   Max.   :594.00  
    fandango         cntwait      
 Min.   :  35.0   Min.   :0.1500  
 1st Qu.: 254.8   1st Qu.:0.3600  
 Median : 430.5   Median :0.4850  
 Mean   : 522.3   Mean   :0.4824  
 3rd Qu.: 663.5   3rd Qu.:0.5875  
 Max.   :1778.0   Max.   :0.7900  
    \end{Verbatim}

    
    0

    
    \begin{tabular}{r|lllllllllllll}
 box & mprating & budget & starpower & sequel & action & comedy & animated & horror & addict & cmngsoon & fandango & cntwait\\
\hline
	 191.67085 & 4         &  28.0     & 19.83     & FALSE     & FALSE     & TRUE      & FALSE     & FALSE     & 7860.5    & 10        & 144       & 0.49     \\
	 631.06589 & 2         & 150.0     & 32.69     & TRUE      & FALSE     & FALSE     & TRUE      & FALSE     & 5737.0    & 59        & 468       & 0.79     \\
	  54.01605 & 4         &  37.4     & 15.69     & FALSE     & FALSE     & TRUE      & FALSE     & FALSE     &  850.0    & 24        & 198       & 0.36     \\
\end{tabular}


    
    We see that 14 over the graph indicate the number of covariates entering
the model as \$ \lambda\$ varies: 14 is repeated, as ridge regression is
not a selection method.

    \begin{tcolorbox}[breakable, size=fbox, boxrule=1pt, pad at break*=1mm,colback=cellbackground, colframe=cellborder]
\prompt{In}{incolor}{122}{\boxspacing}
\begin{Verbatim}[commandchars=\\\{\}]
\PY{n+nf}{library}\PY{p}{(}\PY{n}{glmnet}\PY{p}{)}
\PY{n}{y} \PY{o}{\PYZlt{}\PYZhy{}}  \PY{n}{mydata}\PY{o}{\PYZdl{}}\PY{n}{box}
\PY{n}{X} \PY{o}{\PYZlt{}\PYZhy{}} \PY{n+nf}{model.matrix}\PY{p}{(}\PY{n}{box} \PY{o}{\PYZti{}} \PY{n}{.}\PY{p}{,} \PY{n}{data}\PY{o}{=}\PY{n}{mydata}\PY{p}{)}\PY{p}{[}\PY{p}{,}\PY{l+m}{\PYZhy{}1}\PY{p}{]}
\PY{n}{m.ridge} \PY{o}{\PYZlt{}\PYZhy{}} \PY{n+nf}{glmnet}\PY{p}{(}\PY{n}{X}\PY{p}{,} \PY{n}{y}\PY{p}{,} \PY{n}{alpha}\PY{o}{=}\PY{l+m}{0}\PY{p}{)}

\PY{c+c1}{\PYZsh{}\PYZsh{} plot graphical evaluation}

\PY{n+nf}{options}\PY{p}{(}\PY{n}{repr.plot.width} \PY{o}{=} \PY{l+m}{15}\PY{p}{,} \PY{n}{repr.plot.height} \PY{o}{=} \PY{l+m}{7}\PY{p}{)}
\PY{n+nf}{par}\PY{p}{(}\PY{n}{mar} \PY{o}{=} \PY{n+nf}{c}\PY{p}{(}\PY{l+m}{5.1}\PY{p}{,} \PY{l+m}{6.1}\PY{p}{,} \PY{l+m}{4.1}\PY{p}{,} \PY{l+m}{2.1}\PY{p}{)}\PY{p}{)}
    \PY{n+nf}{par}\PY{p}{(}\PY{n}{mfrow}\PY{o}{=}\PY{n+nf}{c}\PY{p}{(}\PY{l+m}{1}\PY{p}{,}\PY{l+m}{2}\PY{p}{)}\PY{p}{)}

\PY{n+nf}{plot}\PY{p}{(}\PY{n}{m.ridge}\PY{p}{,} \PY{n}{xvar}\PY{o}{=}\PY{l+s}{\PYZsq{}}\PY{l+s}{lambda\PYZsq{}}\PY{p}{,} \PY{n}{xlab}\PY{o}{=}\PY{n+nf}{expression}\PY{p}{(}\PY{n+nf}{log}\PY{p}{(}\PY{n}{lambda}\PY{p}{)}\PY{p}{)}\PY{p}{)}
\PY{n+nf}{title}\PY{p}{(}\PY{n+nf}{expression}\PY{p}{(}\PY{n}{Graphical} \PY{o}{\PYZti{}} \PY{n}{evaluation} \PY{o}{\PYZti{}}\PY{n}{of}\PY{o}{\PYZti{}} \PY{n+nf}{symbol}\PY{p}{(}\PY{n}{l}\PY{p}{)}\PY{o}{\PYZti{}}\PY{n}{with}\PY{o}{\PYZti{}}\PY{n}{x}\PY{p}{[}\PY{n}{axis}\PY{p}{]}\PY{o}{\PYZti{}}\PY{n}{expressed}\PY{o}{\PYZti{}}\PY{n}{In}\PY{o}{\PYZti{}}\PY{n}{terms}\PY{o}{\PYZti{}}\PY{n}{of}\PY{o}{\PYZti{}}\PY{n+nf}{symbol}\PY{p}{(}\PY{n}{l}\PY{p}{)}\PY{p}{)}\PY{p}{,}\PY{n}{line}\PY{o}{=}\PY{l+m}{2.7}\PY{p}{)}
\PY{n+nf}{plot}\PY{p}{(}\PY{n}{m.ridge}\PY{p}{,} \PY{n}{xlab}\PY{o}{=}\PY{n+nf}{expression}\PY{p}{(}\PY{n+nf}{log}\PY{p}{(}\PY{n}{lambda}\PY{p}{)}\PY{p}{)}\PY{p}{)}
\PY{n+nf}{title}\PY{p}{(}\PY{n+nf}{expression}\PY{p}{(}\PY{n}{Graphical} \PY{o}{\PYZti{}} \PY{n}{evaluation} \PY{o}{\PYZti{}}\PY{n}{of}\PY{o}{\PYZti{}} \PY{n+nf}{symbol}\PY{p}{(}\PY{n}{l}\PY{p}{)}\PY{o}{\PYZti{}}\PY{n}{without}\PY{o}{\PYZti{}}\PY{n}{x}\PY{p}{[}\PY{n}{axis}\PY{p}{]}\PY{o}{\PYZti{}}\PY{n}{expressed}\PY{o}{\PYZti{}}\PY{n}{In}\PY{o}{\PYZti{}}\PY{n}{terms}\PY{o}{\PYZti{}}\PY{n}{of}\PY{o}{\PYZti{}}\PY{n+nf}{symbol}\PY{p}{(}\PY{n}{l}\PY{p}{)}\PY{p}{)}\PY{p}{,}\PY{n}{line}\PY{o}{=}\PY{l+m}{2.7}\PY{p}{)}
\end{Verbatim}
\end{tcolorbox}

    \begin{center}
    \adjustimage{max size={0.9\linewidth}{0.9\paperheight}}{output_26_0.png}
    \end{center}
    { \hspace*{\fill} \\}
    
    Now let's look for the best \(\lambda\) using cross validation. The plot
below shows the values of \(cvm\) for each \(log(\lambda)\) together
with the associated confidence interval. The two dashed lines are the
values of minimun \(log(\lambda)\) and \(log(\lambda)\) \(1 \sigma\) far
from the minimum.\\
So the best \(\lambda\) from cross validation is: 71\\
And the MSE is: 16003

    \begin{tcolorbox}[breakable, size=fbox, boxrule=1pt, pad at break*=1mm,colback=cellbackground, colframe=cellborder]
\prompt{In}{incolor}{126}{\boxspacing}
\begin{Verbatim}[commandchars=\\\{\}]
\PY{c+c1}{\PYZsh{}\PYZsh{}\PYZsh{} cross validation plot}

\PY{n+nf}{set.seed}\PY{p}{(}\PY{l+m}{2906}\PY{p}{)}
\PY{n+nf}{options}\PY{p}{(}\PY{n}{warn}\PY{o}{=}\PY{l+m}{\PYZhy{}1}\PY{p}{)}
\PY{n}{cv.ridge} \PY{o}{\PYZlt{}\PYZhy{}} \PY{n+nf}{cv.glmnet}\PY{p}{(}\PY{n}{X}\PY{p}{,} \PY{n}{y}\PY{p}{,} \PY{n}{alpha}\PY{o}{=}\PY{l+m}{0}\PY{p}{)}
\PY{n+nf}{plot}\PY{p}{(}\PY{n}{cv.ridge}\PY{p}{,} \PY{n}{xlab}\PY{o}{=}\PY{n+nf}{expression}\PY{p}{(}\PY{n+nf}{log}\PY{p}{(}\PY{n}{lambda}\PY{p}{)}\PY{p}{)}\PY{p}{)}
\PY{n+nf}{text}\PY{p}{(}\PY{n}{x}\PY{o}{=}\PY{n+nf}{log}\PY{p}{(}\PY{n}{cv.ridge}\PY{o}{\PYZdl{}}\PY{n}{lambda.min}\PY{p}{)}\PY{l+m}{+0.5}\PY{p}{,} \PY{n}{y}\PY{o}{=}\PY{l+m}{30000}\PY{p}{,} \PY{n+nf}{paste0}\PY{p}{(}\PY{l+s}{\PYZdq{}}\PY{l+s}{ best log(λ) =\PYZdq{}}\PY{p}{,}\PY{n+nf}{round}\PY{p}{(}\PY{p}{(}\PY{n+nf}{log}\PY{p}{(}\PY{n}{cv.ridge}\PY{o}{\PYZdl{}}\PY{n}{lambda.min}\PY{p}{)}\PY{p}{)}\PY{p}{,}\PY{l+m}{4}\PY{p}{)}\PY{p}{)}\PY{p}{)}
\PY{n+nf}{title}\PY{p}{(}\PY{n+nf}{expression}\PY{p}{(}\PY{n}{cross} \PY{o}{\PYZti{}} \PY{n}{validation} \PY{o}{\PYZti{}}\PY{n}{For}\PY{o}{\PYZti{}} \PY{n+nf}{symbol}\PY{p}{(}\PY{n}{l}\PY{p}{)}\PY{p}{)}\PY{p}{,}\PY{n}{line}\PY{o}{=}\PY{l+m}{2.7}\PY{p}{)}

\PY{c+c1}{\PYZsh{}\PYZsh{} best lambda and MSE}

\PY{n}{best.lambda} \PY{o}{\PYZlt{}\PYZhy{}} \PY{n}{cv.ridge}\PY{o}{\PYZdl{}}\PY{n}{lambda.min}
\PY{n}{best.lambda}

\PY{n}{mse}\PY{o}{\PYZlt{}\PYZhy{}}\PY{n+nf}{min}\PY{p}{(}\PY{n}{cv.ridge}\PY{o}{\PYZdl{}}\PY{n}{cvm}\PY{p}{)}
\PY{n}{mse}
\end{Verbatim}
\end{tcolorbox}

    71.2857129477994

    
    16002.6055378586

    
    \begin{center}
    \adjustimage{max size={0.9\linewidth}{0.9\paperheight}}{output_28_2.png}
    \end{center}
    { \hspace*{\fill} \\}
    
    Now we can Re-estimate the model using the best \(\lambda\). Below we
seen the coefficients of the model, graphical representation of the
coefficients for the best \(\lambda\) and model deviance.\\
The maximum explained deviance is obtained for the minimum (best)
\(\lambda\) and it is equal to:0.61

    \begin{tcolorbox}[breakable, size=fbox, boxrule=1pt, pad at break*=1mm,colback=cellbackground, colframe=cellborder]
\prompt{In}{incolor}{125}{\boxspacing}
\begin{Verbatim}[commandchars=\\\{\}]
\PY{c+c1}{\PYZsh{}\PYZsh{} best model}

\PY{n}{m.ridge.min} \PY{o}{\PYZlt{}\PYZhy{}} \PY{n+nf}{glmnet}\PY{p}{(}\PY{n}{X}\PY{p}{,} \PY{n}{y}\PY{p}{,} \PY{n}{alpha}\PY{o}{=}\PY{l+m}{0}\PY{p}{,} \PY{n}{lambda}\PY{o}{=}\PY{n}{best.lambda}\PY{p}{)}

\PY{n+nf}{par}\PY{p}{(}\PY{n}{mfrow}\PY{o}{=}\PY{n+nf}{c}\PY{p}{(}\PY{l+m}{1}\PY{p}{,}\PY{l+m}{2}\PY{p}{)}\PY{p}{)}
\PY{n+nf}{plot}\PY{p}{(}\PY{n}{m.ridge}\PY{p}{,} \PY{n}{xvar}\PY{o}{=}\PY{l+s}{\PYZsq{}}\PY{l+s}{lambda\PYZsq{}}\PY{p}{,} \PY{n}{xlab}\PY{o}{=}\PY{n+nf}{expression}\PY{p}{(}\PY{n+nf}{log}\PY{p}{(}\PY{n}{lambda}\PY{p}{)}\PY{p}{)}\PY{p}{)} \PY{c+c1}{\PYZsh{}\PYZsh{} add on the line corresponding to the best lambda abline(v=log(best.lambda), lty=2)}
\PY{n+nf}{title}\PY{p}{(}\PY{n+nf}{expression}\PY{p}{(}\PY{n}{Graphical} \PY{o}{\PYZti{}} \PY{n}{evaluation} \PY{o}{\PYZti{}}\PY{n}{of}\PY{o}{\PYZti{}} \PY{n+nf}{symbol}\PY{p}{(}\PY{n}{l}\PY{p}{)}\PY{o}{\PYZti{}}\PY{n}{with}\PY{o}{\PYZti{}}\PY{n}{x}\PY{p}{[}\PY{n}{axis}\PY{p}{]}\PY{o}{\PYZti{}}\PY{n}{expressed}\PY{o}{\PYZti{}}\PY{n}{In}\PY{o}{\PYZti{}}\PY{n}{terms}\PY{o}{\PYZti{}}\PY{n}{of}\PY{o}{\PYZti{}}\PY{n+nf}{symbol}\PY{p}{(}\PY{n}{l}\PY{p}{)}\PY{p}{)}\PY{p}{,}\PY{n}{line}\PY{o}{=}\PY{l+m}{2.7}\PY{p}{)}

\PY{c+c1}{\PYZsh{}\PYZsh{} deviance}
\PY{n+nf}{plot}\PY{p}{(}\PY{n+nf}{log}\PY{p}{(}\PY{n}{m.ridge}\PY{o}{\PYZdl{}}\PY{n}{lambda}\PY{p}{)}\PY{p}{,} \PY{n}{m.ridge}\PY{o}{\PYZdl{}}\PY{n}{dev.ratio}\PY{p}{,} \PY{n}{type}\PY{o}{=}\PY{l+s}{\PYZsq{}}\PY{l+s}{l\PYZsq{}}\PY{p}{,}
       \PY{n}{xlab}\PY{o}{=}\PY{n+nf}{expression}\PY{p}{(}\PY{n+nf}{log}\PY{p}{(}\PY{n}{lambda}\PY{p}{)}\PY{p}{)}\PY{p}{,} \PY{n}{ylab}\PY{o}{=}\PY{l+s}{\PYZsq{}}\PY{l+s}{Explained deviance\PYZsq{}}\PY{p}{)}
\PY{n+nf}{abline}\PY{p}{(}\PY{n}{v}\PY{o}{=}\PY{n+nf}{log}\PY{p}{(}\PY{n}{best.lambda}\PY{p}{)}\PY{p}{,} \PY{n}{lty}\PY{o}{=}\PY{l+m}{2}\PY{p}{,}\PY{n}{col}\PY{o}{=}\PY{l+s}{\PYZdq{}}\PY{l+s}{blue\PYZdq{}}\PY{p}{)}
\PY{n+nf}{legend}\PY{p}{(}\PY{n}{legend} \PY{o}{=} \PY{n+nf}{expression}\PY{p}{(}\PY{n}{best}\PY{o}{\PYZti{}} \PY{n+nf}{symbol}\PY{p}{(}\PY{n}{l}\PY{p}{)}\PY{p}{)}\PY{p}{,} \PY{n}{col}\PY{o}{=}\PY{l+s}{\PYZdq{}}\PY{l+s}{blue\PYZdq{}}\PY{p}{,}\PY{n}{lt}\PY{o}{=}\PY{l+m}{2}\PY{p}{,}\PY{n}{x} \PY{o}{=} \PY{l+m}{2.5}\PY{p}{,}\PY{n}{bty}\PY{o}{=}\PY{l+s}{\PYZdq{}}\PY{l+s}{n\PYZdq{}}\PY{p}{,} \PY{n}{y}\PY{o}{=}\PY{l+m}{0.5}\PY{p}{)}
\PY{n+nf}{title }\PY{p}{(}\PY{l+s}{\PYZdq{}}\PY{l+s}{Graph of explained deviance\PYZdq{}}\PY{p}{,}\PY{n}{line}\PY{o}{=}\PY{l+m}{2.7}\PY{p}{)}


\PY{c+c1}{\PYZsh{}\PYZsh{} maxi explained deviance}
\end{Verbatim}
\end{tcolorbox}

    \begin{center}
    \adjustimage{max size={0.9\linewidth}{0.9\paperheight}}{output_30_0.png}
    \end{center}
    { \hspace*{\fill} \\}
    
    \hypertarget{lasso}{%
\subsubsection{LASSO}\label{lasso}}

Let's perform the analysis usign lasso.

Above we can see the graphical evaluation of the coefficients associated
to the covariates. We see that 14 over the graph indicate the number of
covariates entering the model as \$ \lambda\$ varies:14 is not repeated,
as lasso regression is a selection method.

    \begin{tcolorbox}[breakable, size=fbox, boxrule=1pt, pad at break*=1mm,colback=cellbackground, colframe=cellborder]
\prompt{In}{incolor}{127}{\boxspacing}
\begin{Verbatim}[commandchars=\\\{\}]
\PY{c+c1}{\PYZsh{}\PYZsh{} lasso regression}
\PY{n+nf}{library}\PY{p}{(}\PY{n}{glmnet}\PY{p}{)}
\PY{n}{m.lasso} \PY{o}{\PYZlt{}\PYZhy{}} \PY{n+nf}{glmnet}\PY{p}{(}\PY{n}{X}\PY{p}{,} \PY{n}{y}\PY{p}{,} \PY{n}{alpha}\PY{o}{=}\PY{l+m}{1}\PY{p}{)}

\PY{n+nf}{options}\PY{p}{(}\PY{n}{repr.plot.width} \PY{o}{=} \PY{l+m}{15}\PY{p}{,} \PY{n}{repr.plot.height} \PY{o}{=} \PY{l+m}{7}\PY{p}{)}
\PY{n+nf}{par}\PY{p}{(}\PY{n}{mar} \PY{o}{=} \PY{n+nf}{c}\PY{p}{(}\PY{l+m}{5.1}\PY{p}{,} \PY{l+m}{6.1}\PY{p}{,} \PY{l+m}{4.1}\PY{p}{,} \PY{l+m}{2.1}\PY{p}{)}\PY{p}{)}
    \PY{n+nf}{par}\PY{p}{(}\PY{n}{mfrow}\PY{o}{=}\PY{n+nf}{c}\PY{p}{(}\PY{l+m}{1}\PY{p}{,}\PY{l+m}{2}\PY{p}{)}\PY{p}{)}

\PY{n+nf}{plot}\PY{p}{(}\PY{n}{m.lasso}\PY{p}{,} \PY{n}{xvar}\PY{o}{=}\PY{l+s}{\PYZsq{}}\PY{l+s}{lambda\PYZsq{}}\PY{p}{,} \PY{n}{xlab}\PY{o}{=}\PY{n+nf}{expression}\PY{p}{(}\PY{n+nf}{log}\PY{p}{(}\PY{n}{lambda}\PY{p}{)}\PY{p}{)}\PY{p}{)}
\PY{n+nf}{title}\PY{p}{(}\PY{n+nf}{expression}\PY{p}{(}\PY{n}{Graphical} \PY{o}{\PYZti{}} \PY{n}{evaluation} \PY{o}{\PYZti{}}\PY{n}{of}\PY{o}{\PYZti{}} \PY{n+nf}{symbol}\PY{p}{(}\PY{n}{l}\PY{p}{)}\PY{o}{\PYZti{}}\PY{n}{with}\PY{o}{\PYZti{}}\PY{n}{x}\PY{p}{[}\PY{n}{axis}\PY{p}{]}\PY{o}{\PYZti{}}\PY{n}{expressed}\PY{o}{\PYZti{}}\PY{n}{In}\PY{o}{\PYZti{}}\PY{n}{terms}\PY{o}{\PYZti{}}\PY{n}{of}\PY{o}{\PYZti{}}\PY{n+nf}{symbol}\PY{p}{(}\PY{n}{l}\PY{p}{)}\PY{p}{)}\PY{p}{,}\PY{n}{line}\PY{o}{=}\PY{l+m}{2.7}\PY{p}{)}
\PY{n+nf}{plot}\PY{p}{(}\PY{n}{m.lasso}\PY{p}{,} \PY{n}{xlab}\PY{o}{=}\PY{n+nf}{expression}\PY{p}{(}\PY{n+nf}{log}\PY{p}{(}\PY{n}{lambda}\PY{p}{)}\PY{p}{)}\PY{p}{)}
\PY{n+nf}{title}\PY{p}{(}\PY{n+nf}{expression}\PY{p}{(}\PY{n}{Graphical} \PY{o}{\PYZti{}} \PY{n}{evaluation} \PY{o}{\PYZti{}}\PY{n}{of}\PY{o}{\PYZti{}} \PY{n+nf}{symbol}\PY{p}{(}\PY{n}{l}\PY{p}{)}\PY{o}{\PYZti{}}\PY{n}{without}\PY{o}{\PYZti{}}\PY{n}{x}\PY{p}{[}\PY{n}{axis}\PY{p}{]}\PY{o}{\PYZti{}}\PY{n}{expressed}\PY{o}{\PYZti{}}\PY{n}{In}\PY{o}{\PYZti{}}\PY{n}{terms}\PY{o}{\PYZti{}}\PY{n}{of}\PY{o}{\PYZti{}}\PY{n+nf}{symbol}\PY{p}{(}\PY{n}{l}\PY{p}{)}\PY{p}{)}\PY{p}{,}\PY{n}{line}\PY{o}{=}\PY{l+m}{2.7}\PY{p}{)}
\end{Verbatim}
\end{tcolorbox}

    \begin{center}
    \adjustimage{max size={0.9\linewidth}{0.9\paperheight}}{output_32_0.png}
    \end{center}
    { \hspace*{\fill} \\}
    
    Now let's look for the best \(\lambda\) using cross validation. So the
best \(\lambda\) from cross validation is: 5\\
And the MSE is: 16201

    \begin{tcolorbox}[breakable, size=fbox, boxrule=1pt, pad at break*=1mm,colback=cellbackground, colframe=cellborder]
\prompt{In}{incolor}{131}{\boxspacing}
\begin{Verbatim}[commandchars=\\\{\}]
\PY{n+nf}{set.seed}\PY{p}{(}\PY{l+m}{2906}\PY{p}{)}
\PY{n+nf}{options}\PY{p}{(}\PY{n}{warn}\PY{o}{=}\PY{l+m}{\PYZhy{}1}\PY{p}{)}
\PY{n}{cv.lasso} \PY{o}{\PYZlt{}\PYZhy{}} \PY{n+nf}{cv.glmnet}\PY{p}{(}\PY{n}{X}\PY{p}{,} \PY{n}{y}\PY{p}{,} \PY{n}{alpha}\PY{o}{=}\PY{l+m}{1}\PY{p}{)}
\PY{n+nf}{plot}\PY{p}{(}\PY{n}{cv.lasso}\PY{p}{,} \PY{n}{xlab}\PY{o}{=}\PY{n+nf}{expression}\PY{p}{(}\PY{n+nf}{log}\PY{p}{(}\PY{n}{lambda}\PY{p}{)}\PY{p}{)}\PY{p}{)}
\PY{n+nf}{text}\PY{p}{(}\PY{n}{x}\PY{o}{=}\PY{n+nf}{log}\PY{p}{(}\PY{n}{cv.lasso}\PY{o}{\PYZdl{}}\PY{n}{lambda.min}\PY{p}{)}\PY{l+m}{+0.5}\PY{p}{,} \PY{n}{y}\PY{o}{=}\PY{l+m}{30000}\PY{p}{,} \PY{n+nf}{paste0}\PY{p}{(}\PY{l+s}{\PYZdq{}}\PY{l+s}{ best log(λ) =\PYZdq{}}\PY{p}{,}\PY{n+nf}{round}\PY{p}{(}\PY{p}{(}\PY{n+nf}{log}\PY{p}{(}\PY{n}{cv.lasso}\PY{o}{\PYZdl{}}\PY{n}{lambda.min}\PY{p}{)}\PY{p}{)}\PY{p}{,}\PY{l+m}{4}\PY{p}{)}\PY{p}{)}\PY{p}{)}
\PY{n+nf}{title}\PY{p}{(}\PY{n+nf}{expression}\PY{p}{(}\PY{n}{cross} \PY{o}{\PYZti{}} \PY{n}{validation} \PY{o}{\PYZti{}}\PY{n}{For}\PY{o}{\PYZti{}} \PY{n+nf}{symbol}\PY{p}{(}\PY{n}{l}\PY{p}{)}\PY{p}{)}\PY{p}{,}\PY{n}{line}\PY{o}{=}\PY{l+m}{2.7}\PY{p}{)}

\PY{c+c1}{\PYZsh{}\PYZsh{} best lambda and MSE}

\PY{n}{best.lambda} \PY{o}{\PYZlt{}\PYZhy{}} \PY{n}{cv.lasso}\PY{o}{\PYZdl{}}\PY{n}{lambda.min}


\PY{n}{mse}\PY{o}{\PYZlt{}\PYZhy{}}\PY{n+nf}{min}\PY{p}{(}\PY{n}{cv.lasso}\PY{o}{\PYZdl{}}\PY{n}{cvm}\PY{p}{)}
\end{Verbatim}
\end{tcolorbox}

    \begin{center}
    \adjustimage{max size={0.9\linewidth}{0.9\paperheight}}{output_34_0.png}
    \end{center}
    { \hspace*{\fill} \\}
    
    On the basis of MSE , the model fitted with lasso has got the same MSE
by the way The resulting model with lasso is simpler. Now we can
Re-estimate the model using the best \(\lambda\). Below we seen the
coefficients of the model, graphical representation of the coefficients
for the best \(\lambda\) and model deviance.\\
The maximum explained deviance is obtained for the minimum (best)
\(\lambda\) and it is equal to: 0.72\\
Furthermore from the new coefficients we can see that some of the
coefficients are zero, so the lasso performed a model selection. In
particular thenot coefficients equal to 0 are: comedytrue, animatedtrue
, fandango, and starpower. Also mprating3 is set to 0. So the total
number of coefficients different from 0 is 10.

    \begin{tcolorbox}[breakable, size=fbox, boxrule=1pt, pad at break*=1mm,colback=cellbackground, colframe=cellborder]
\prompt{In}{incolor}{135}{\boxspacing}
\begin{Verbatim}[commandchars=\\\{\}]
\PY{c+c1}{\PYZsh{}\PYZsh{}\PYZsh{}\PYZsh{} best model}

\PY{n}{m.lasso.min} \PY{o}{\PYZlt{}\PYZhy{}} \PY{n+nf}{glmnet}\PY{p}{(}\PY{n}{X}\PY{p}{,} \PY{n}{y}\PY{p}{,} \PY{n}{alpha}\PY{o}{=}\PY{l+m}{1}\PY{p}{,} \PY{n}{lambda}\PY{o}{=}\PY{n}{best.lambda}\PY{p}{)}
\PY{n}{m.lasso.min}
\PY{n+nf}{coef}\PY{p}{(}\PY{n}{m.lasso.min}\PY{p}{)}

\PY{c+c1}{\PYZsh{}\PYZsh{} number of coefficient diversi da 0}

\PY{n}{id.zero} \PY{o}{\PYZlt{}\PYZhy{}} \PY{n+nf}{which}\PY{p}{(}\PY{n+nf}{coef}\PY{p}{(}\PY{n}{m.lasso.min}\PY{p}{)}\PY{o}{==}\PY{l+m}{0}\PY{p}{)}
\PY{n}{nonzero} \PY{o}{\PYZlt{}\PYZhy{}} \PY{n+nf}{length}\PY{p}{(}\PY{n+nf}{coef}\PY{p}{(}\PY{n}{m.lasso.min}\PY{p}{)}\PY{p}{)}\PY{o}{\PYZhy{}}\PY{n+nf}{length}\PY{p}{(}\PY{n}{id.zero}\PY{p}{)}

\PY{n}{nonzero}


\PY{n+nf}{par}\PY{p}{(}\PY{n}{mfrow}\PY{o}{=}\PY{n+nf}{c}\PY{p}{(}\PY{l+m}{1}\PY{p}{,}\PY{l+m}{2}\PY{p}{)}\PY{p}{)}
\PY{n+nf}{plot}\PY{p}{(}\PY{n}{m.lasso}\PY{p}{,} \PY{n}{xvar}\PY{o}{=}\PY{l+s}{\PYZsq{}}\PY{l+s}{lambda\PYZsq{}}\PY{p}{,} \PY{n}{xlab}\PY{o}{=}\PY{n+nf}{expression}\PY{p}{(}\PY{n+nf}{log}\PY{p}{(}\PY{n}{lambda}\PY{p}{)}\PY{p}{)}\PY{p}{)} \PY{c+c1}{\PYZsh{}\PYZsh{} add on the line corresponding to the best lambda abline(v=log(best.lambda), lty=2)}
\PY{n+nf}{title}\PY{p}{(}\PY{n+nf}{expression}\PY{p}{(}\PY{n}{Graphical} \PY{o}{\PYZti{}} \PY{n}{evaluation} \PY{o}{\PYZti{}}\PY{n}{of}\PY{o}{\PYZti{}} \PY{n+nf}{symbol}\PY{p}{(}\PY{n}{l}\PY{p}{)}\PY{o}{\PYZti{}}\PY{n}{with}\PY{o}{\PYZti{}}\PY{n}{x}\PY{p}{[}\PY{n}{axis}\PY{p}{]}\PY{o}{\PYZti{}}\PY{n}{expressed}\PY{o}{\PYZti{}}\PY{n}{In}\PY{o}{\PYZti{}}\PY{n}{terms}\PY{o}{\PYZti{}}\PY{n}{of}\PY{o}{\PYZti{}}\PY{n+nf}{symbol}\PY{p}{(}\PY{n}{l}\PY{p}{)}\PY{p}{)}\PY{p}{,}\PY{n}{line}\PY{o}{=}\PY{l+m}{2.7}\PY{p}{)}

\PY{c+c1}{\PYZsh{}\PYZsh{} deviance}
\PY{n+nf}{plot}\PY{p}{(}\PY{n+nf}{log}\PY{p}{(}\PY{n}{m.lasso}\PY{o}{\PYZdl{}}\PY{n}{lambda}\PY{p}{)}\PY{p}{,} \PY{n}{m.lasso}\PY{o}{\PYZdl{}}\PY{n}{dev.ratio}\PY{p}{,} \PY{n}{type}\PY{o}{=}\PY{l+s}{\PYZsq{}}\PY{l+s}{l\PYZsq{}}\PY{p}{,}
        \PY{n}{xlab}\PY{o}{=}\PY{n+nf}{expression}\PY{p}{(}\PY{n+nf}{log}\PY{p}{(}\PY{n}{lambda}\PY{p}{)}\PY{p}{)}\PY{p}{,} \PY{n}{ylab}\PY{o}{=}\PY{l+s}{\PYZsq{}}\PY{l+s}{Explained deviance\PYZsq{}}\PY{p}{)}
\PY{n+nf}{abline}\PY{p}{(}\PY{n}{v}\PY{o}{=}\PY{n+nf}{log}\PY{p}{(}\PY{n}{best.lambda}\PY{p}{)}\PY{p}{,} \PY{n}{lty}\PY{o}{=}\PY{l+m}{2}\PY{p}{,}\PY{n}{col}\PY{o}{=}\PY{l+s}{\PYZdq{}}\PY{l+s}{blue\PYZdq{}}\PY{p}{)}
\PY{n+nf}{legend}\PY{p}{(}\PY{n}{legend} \PY{o}{=} \PY{n+nf}{expression}\PY{p}{(}\PY{n}{best}\PY{o}{\PYZti{}} \PY{n+nf}{symbol}\PY{p}{(}\PY{n}{l}\PY{p}{)}\PY{p}{)}\PY{p}{,} \PY{n}{col}\PY{o}{=}\PY{l+s}{\PYZdq{}}\PY{l+s}{blue\PYZdq{}}\PY{p}{,}\PY{n}{lt}\PY{o}{=}\PY{l+m}{2}\PY{p}{,}\PY{n}{x} \PY{o}{=} \PY{l+m}{\PYZhy{}6}\PY{p}{,}\PY{n}{bty}\PY{o}{=}\PY{l+s}{\PYZdq{}}\PY{l+s}{n\PYZdq{}}\PY{p}{,} \PY{n}{y}\PY{o}{=}\PY{l+m}{0.5}\PY{p}{)}
\PY{n+nf}{title }\PY{p}{(}\PY{l+s}{\PYZdq{}}\PY{l+s}{Graph of explained deviance\PYZdq{}}\PY{p}{,}\PY{n}{line}\PY{o}{=}\PY{l+m}{2.7}\PY{p}{)}

\PY{c+c1}{\PYZsh{}\PYZsh{} maxi explained deviance}
\PY{n+nf}{max}\PY{p}{(}\PY{n}{m.lasso}\PY{o}{\PYZdl{}}\PY{n}{dev.ratio}\PY{p}{)}
\end{Verbatim}
\end{tcolorbox}

    
    \begin{Verbatim}[commandchars=\\\{\}]

Call:  glmnet(x = X, y = y, alpha = 1, lambda = best.lambda) 

     Df   \%Dev Lambda
[1,]  9 0.7104  5.147
    \end{Verbatim}

    
    
    \begin{Verbatim}[commandchars=\\\{\}]
15 x 1 sparse Matrix of class "dgCMatrix"
                        s0
(Intercept)  -46.209355470
mprating2     32.812449674
mprating3      .          
mprating4    -25.156457342
budget         1.163192402
starpower      .          
sequelTRUE    92.792137770
actionTRUE   -97.469669173
comedyTRUE     .          
animatedTRUE   .          
horrorTRUE     3.672794196
addict         0.006753868
cmngsoon       0.298359645
fandango       .          
cntwait      282.999782870
    \end{Verbatim}

    
    10

    
    0.722200254013464

    
    \begin{center}
    \adjustimage{max size={0.9\linewidth}{0.9\paperheight}}{output_36_4.png}
    \end{center}
    { \hspace*{\fill} \\}
    
    Compare the results with those from the linear model. We have that : -
MSE for lasso is:16201 - MSE for linear model is:16332

There is a difference. Since there is substantial variable selection,
lasso is more interesting and we keep it.

Let's also plot the prediction for both methods.

    \begin{tcolorbox}[breakable, size=fbox, boxrule=1pt, pad at break*=1mm,colback=cellbackground, colframe=cellborder]
\prompt{In}{incolor}{138}{\boxspacing}
\begin{Verbatim}[commandchars=\\\{\}]
\PY{n}{prediction.ridge} \PY{o}{\PYZlt{}\PYZhy{}} \PY{n+nf}{predict}\PY{p}{(}\PY{n}{m.ridge.min}\PY{p}{,} \PY{n}{newx}\PY{o}{=}\PY{n}{X}\PY{p}{)}
\PY{n}{prediction.lasso} \PY{o}{\PYZlt{}\PYZhy{}} \PY{n+nf}{predict}\PY{p}{(}\PY{n}{m.lasso.min}\PY{p}{,} \PY{n}{newx}\PY{o}{=}\PY{n}{X}\PY{p}{)}
\PY{n+nf}{par}\PY{p}{(}\PY{n}{mfrow}\PY{o}{=}\PY{n+nf}{c}\PY{p}{(}\PY{l+m}{1}\PY{p}{,}\PY{l+m}{2}\PY{p}{)}\PY{p}{)}
\PY{n+nf}{plot}\PY{p}{(}\PY{n}{prediction.ridge}\PY{p}{,} \PY{n}{mydata}\PY{o}{\PYZdl{}}\PY{n}{box}\PY{p}{,}
        \PY{n}{main}\PY{o}{=}\PY{l+s}{\PYZsq{}}\PY{l+s}{Predictions with Ridge\PYZsq{}}\PY{p}{,}\PY{n}{col}\PY{o}{=}\PY{l+s}{\PYZdq{}}\PY{l+s}{blue\PYZdq{}}\PY{p}{,}\PY{n}{pch}\PY{o}{=}\PY{l+m}{19}\PY{p}{)}
\PY{n+nf}{abline}\PY{p}{(}\PY{l+m}{0}\PY{p}{,}\PY{l+m}{1}\PY{p}{,}\PY{n}{col}\PY{o}{=}\PY{l+s}{\PYZdq{}}\PY{l+s}{red\PYZdq{}}\PY{p}{)}
\PY{n+nf}{plot}\PY{p}{(}\PY{n}{prediction.lasso}\PY{p}{,}\PY{n}{mydata}\PY{o}{\PYZdl{}}\PY{n}{box}\PY{p}{,}
        \PY{n}{main}\PY{o}{=}\PY{l+s}{\PYZsq{}}\PY{l+s}{Predictions with Lasso\PYZsq{}}\PY{p}{,}\PY{n}{col}\PY{o}{=}\PY{l+s}{\PYZdq{}}\PY{l+s}{blue\PYZdq{}}\PY{p}{,}\PY{n}{pch}\PY{o}{=}\PY{l+m}{19}\PY{p}{)}
\PY{n+nf}{abline}\PY{p}{(}\PY{l+m}{0}\PY{p}{,}\PY{l+m}{1}\PY{p}{,}\PY{n}{col}\PY{o}{=}\PY{l+s}{\PYZdq{}}\PY{l+s}{red\PYZdq{}}\PY{p}{)}
\end{Verbatim}
\end{tcolorbox}

    \begin{center}
    \adjustimage{max size={0.9\linewidth}{0.9\paperheight}}{output_38_0.png}
    \end{center}
    { \hspace*{\fill} \\}
    
    \hypertarget{automatic-selection}{%
\subsubsection{AUTOMATIC SELECTION}\label{automatic-selection}}

\hypertarget{forward-selection}{%
\paragraph{FORWARD SELECTION}\label{forward-selection}}

Let's try automatic selection

    \begin{tcolorbox}[breakable, size=fbox, boxrule=1pt, pad at break*=1mm,colback=cellbackground, colframe=cellborder]
\prompt{In}{incolor}{156}{\boxspacing}
\begin{Verbatim}[commandchars=\\\{\}]
\PY{n+nf}{library}\PY{p}{(}\PY{n}{leaps}\PY{p}{)}
\PY{n}{m.forward} \PY{o}{\PYZlt{}\PYZhy{}} \PY{n+nf}{regsubsets}\PY{p}{(}\PY{n}{box} \PY{o}{\PYZti{}} \PY{n}{.}\PY{p}{,} \PY{n}{data}\PY{o}{=}\PY{n}{mydata}\PY{p}{,} \PY{n}{nvmax}\PY{o}{=}\PY{l+m}{14}\PY{p}{,} \PY{n}{method}\PY{o}{=}\PY{l+s}{\PYZsq{}}\PY{l+s}{forward\PYZsq{}}\PY{p}{)}
\PY{n+nf}{summary}\PY{p}{(}\PY{n}{m.forward} \PY{p}{)}
\PY{c+c1}{\PYZsh{}\PYZsh{} BIC and RSS}

\PY{c+c1}{\PYZsh{}rss}
\PY{n+nf}{which.min}\PY{p}{(}\PY{n+nf}{summary}\PY{p}{(}\PY{n}{m.forward}\PY{p}{)}\PY{o}{\PYZdl{}}\PY{n}{rss}\PY{p}{)}


\PY{c+c1}{\PYZsh{} BIc}
\PY{n+nf}{which.min}\PY{p}{(}\PY{n+nf}{summary}\PY{p}{(}\PY{n}{m.forward}\PY{p}{)}\PY{o}{\PYZdl{}}\PY{n}{bic}\PY{p}{)}
\end{Verbatim}
\end{tcolorbox}

    
    \begin{Verbatim}[commandchars=\\\{\}]
Subset selection object
Call: regsubsets.formula(box \textasciitilde{} ., data = mydata, nvmax = 14, method = "forward")
14 Variables  (and intercept)
             Forced in Forced out
mprating2        FALSE      FALSE
mprating3        FALSE      FALSE
mprating4        FALSE      FALSE
budget           FALSE      FALSE
starpower        FALSE      FALSE
sequelTRUE       FALSE      FALSE
actionTRUE       FALSE      FALSE
comedyTRUE       FALSE      FALSE
animatedTRUE     FALSE      FALSE
horrorTRUE       FALSE      FALSE
addict           FALSE      FALSE
cmngsoon         FALSE      FALSE
fandango         FALSE      FALSE
cntwait          FALSE      FALSE
1 subsets of each size up to 14
Selection Algorithm: forward
          mprating2 mprating3 mprating4 budget starpower sequelTRUE actionTRUE
1  ( 1 )  " "       " "       " "       " "    " "       " "        " "       
2  ( 1 )  " "       " "       " "       " "    " "       " "        " "       
3  ( 1 )  " "       " "       " "       "*"    " "       " "        " "       
4  ( 1 )  " "       " "       " "       "*"    " "       "*"        " "       
5  ( 1 )  " "       " "       " "       "*"    " "       "*"        "*"       
6  ( 1 )  " "       " "       " "       "*"    " "       "*"        "*"       
7  ( 1 )  "*"       " "       " "       "*"    " "       "*"        "*"       
8  ( 1 )  "*"       "*"       " "       "*"    " "       "*"        "*"       
9  ( 1 )  "*"       "*"       " "       "*"    "*"       "*"        "*"       
10  ( 1 ) "*"       "*"       " "       "*"    "*"       "*"        "*"       
11  ( 1 ) "*"       "*"       " "       "*"    "*"       "*"        "*"       
12  ( 1 ) "*"       "*"       "*"       "*"    "*"       "*"        "*"       
13  ( 1 ) "*"       "*"       "*"       "*"    "*"       "*"        "*"       
14  ( 1 ) "*"       "*"       "*"       "*"    "*"       "*"        "*"       
          comedyTRUE animatedTRUE horrorTRUE addict cmngsoon fandango cntwait
1  ( 1 )  " "        " "          " "        " "    " "      " "      "*"    
2  ( 1 )  " "        " "          " "        "*"    " "      " "      "*"    
3  ( 1 )  " "        " "          " "        "*"    " "      " "      "*"    
4  ( 1 )  " "        " "          " "        "*"    " "      " "      "*"    
5  ( 1 )  " "        " "          " "        "*"    " "      " "      "*"    
6  ( 1 )  " "        " "          " "        "*"    "*"      " "      "*"    
7  ( 1 )  " "        " "          " "        "*"    "*"      " "      "*"    
8  ( 1 )  " "        " "          " "        "*"    "*"      " "      "*"    
9  ( 1 )  " "        " "          " "        "*"    "*"      " "      "*"    
10  ( 1 ) "*"        " "          " "        "*"    "*"      " "      "*"    
11  ( 1 ) "*"        " "          "*"        "*"    "*"      " "      "*"    
12  ( 1 ) "*"        " "          "*"        "*"    "*"      " "      "*"    
13  ( 1 ) "*"        " "          "*"        "*"    "*"      "*"      "*"    
14  ( 1 ) "*"        "*"          "*"        "*"    "*"      "*"      "*"    
    \end{Verbatim}

    
    14

    
    6

    
    From the following plot we see as computed before that the best model
basing on BIC is 6 variables.

    \begin{tcolorbox}[breakable, size=fbox, boxrule=1pt, pad at break*=1mm,colback=cellbackground, colframe=cellborder]
\prompt{In}{incolor}{157}{\boxspacing}
\begin{Verbatim}[commandchars=\\\{\}]
\PY{n+nf}{par}\PY{p}{(}\PY{n}{mfrow}\PY{o}{=}\PY{n+nf}{c}\PY{p}{(}\PY{l+m}{1}\PY{p}{,}\PY{l+m}{2}\PY{p}{)}\PY{p}{)}
\PY{n+nf}{options}\PY{p}{(}\PY{n}{repr.plot.width} \PY{o}{=} \PY{l+m}{15}\PY{p}{,} \PY{n}{repr.plot.height} \PY{o}{=} \PY{l+m}{10}\PY{p}{)}
\PY{n+nf}{plot}\PY{p}{(}\PY{n}{m.forward}\PY{p}{)}
\PY{n+nf}{plot}\PY{p}{(}\PY{n}{m.forward}\PY{p}{,} \PY{n}{scale}\PY{o}{=}\PY{l+s}{\PYZsq{}}\PY{l+s}{adjr2\PYZsq{}}\PY{p}{)}
\end{Verbatim}
\end{tcolorbox}

    \begin{center}
    \adjustimage{max size={0.9\linewidth}{0.9\paperheight}}{output_42_0.png}
    \end{center}
    { \hspace*{\fill} \\}
    
    \begin{tcolorbox}[breakable, size=fbox, boxrule=1pt, pad at break*=1mm,colback=cellbackground, colframe=cellborder]
\prompt{In}{incolor}{158}{\boxspacing}
\begin{Verbatim}[commandchars=\\\{\}]
\PY{n+nf}{par}\PY{p}{(}\PY{n}{mfrow}\PY{o}{=}\PY{n+nf}{c}\PY{p}{(}\PY{l+m}{2}\PY{p}{,}\PY{l+m}{2}\PY{p}{)}\PY{p}{)}
\PY{c+c1}{\PYZsh{}\PYZsh{} R2}
\PY{n+nf}{plot}\PY{p}{(}\PY{n+nf}{summary}\PY{p}{(}\PY{n}{m.forward}\PY{p}{)}\PY{o}{\PYZdl{}}\PY{n}{rsq}\PY{p}{,} \PY{n}{xlab}\PY{o}{=}\PY{l+s}{\PYZsq{}}\PY{l+s}{Number of covariates\PYZsq{}}\PY{p}{,} \PY{n}{ylab}\PY{o}{=}\PY{l+s}{\PYZsq{}}\PY{l+s}{R2\PYZsq{}}\PY{p}{,} \PY{n}{type}\PY{o}{=}\PY{l+s}{\PYZsq{}}\PY{l+s}{l\PYZsq{}}\PY{p}{)}
\PY{c+c1}{\PYZsh{}\PYZsh{} add on the indication of the best model}
\PY{n}{max.rsq} \PY{o}{\PYZlt{}\PYZhy{}} \PY{n+nf}{which.max}\PY{p}{(}\PY{n+nf}{summary}\PY{p}{(}\PY{n}{m.forward}\PY{p}{)}\PY{o}{\PYZdl{}}\PY{n}{rsq}\PY{p}{)}
\PY{n+nf}{points}\PY{p}{(}\PY{n}{max.rsq}\PY{p}{,} \PY{n+nf}{summary}\PY{p}{(}\PY{n}{m.forward}\PY{p}{)}\PY{o}{\PYZdl{}}\PY{n}{rsq}\PY{p}{[}\PY{n}{max.rsq}\PY{p}{]}\PY{p}{,} \PY{n}{col}\PY{o}{=}\PY{l+s}{\PYZsq{}}\PY{l+s}{red\PYZsq{}}\PY{p}{,} \PY{n}{pch}\PY{o}{=}\PY{l+m}{16}\PY{p}{)}
\PY{n+nf}{abline}\PY{p}{(}\PY{n}{v}\PY{o}{=}\PY{n}{max.rsq}\PY{p}{,} \PY{n}{col}\PY{o}{=}\PY{l+s}{\PYZdq{}}\PY{l+s}{blue\PYZdq{}}\PY{p}{,}\PY{n}{lty}\PY{o}{=}\PY{l+m}{2}\PY{p}{)}


\PY{c+c1}{\PYZsh{}\PYZsh{} RSS}
\PY{n+nf}{plot}\PY{p}{(}\PY{n+nf}{summary}\PY{p}{(}\PY{n}{m.forward}\PY{p}{)}\PY{o}{\PYZdl{}}\PY{n}{rss}\PY{p}{,} \PY{n}{xlab}\PY{o}{=}\PY{l+s}{\PYZsq{}}\PY{l+s}{Number of covariates\PYZsq{}}\PY{p}{,} \PY{n}{ylab}\PY{o}{=}\PY{l+s}{\PYZsq{}}\PY{l+s}{RSS\PYZsq{}}\PY{p}{,} \PY{n}{type}\PY{o}{=}\PY{l+s}{\PYZsq{}}\PY{l+s}{l\PYZsq{}}\PY{p}{)}
\PY{n}{min.rss} \PY{o}{\PYZlt{}\PYZhy{}} \PY{n+nf}{which.min}\PY{p}{(}\PY{n+nf}{summary}\PY{p}{(}\PY{n}{m.forward}\PY{p}{)}\PY{o}{\PYZdl{}}\PY{n}{rss}\PY{p}{)}
\PY{n+nf}{points}\PY{p}{(}\PY{n}{min.rss}\PY{p}{,} \PY{n+nf}{summary}\PY{p}{(}\PY{n}{m.forward}\PY{p}{)}\PY{o}{\PYZdl{}}\PY{n}{rss}\PY{p}{[}\PY{n}{min.rss}\PY{p}{]}\PY{p}{,} \PY{n}{col}\PY{o}{=}\PY{l+s}{\PYZsq{}}\PY{l+s}{red\PYZsq{}}\PY{p}{,} \PY{n}{pch}\PY{o}{=}\PY{l+m}{16}\PY{p}{)}
\PY{n+nf}{abline}\PY{p}{(}\PY{n}{v}\PY{o}{=}\PY{n}{min.rss}\PY{p}{,} \PY{n}{col}\PY{o}{=}\PY{l+s}{\PYZdq{}}\PY{l+s}{blue\PYZdq{}}\PY{p}{,}\PY{n}{lty}\PY{o}{=}\PY{l+m}{2}\PY{p}{)}


\PY{c+c1}{\PYZsh{}\PYZsh{} Adjusted R2}
\PY{n+nf}{plot}\PY{p}{(}\PY{n+nf}{summary}\PY{p}{(}\PY{n}{m.forward}\PY{p}{)}\PY{o}{\PYZdl{}}\PY{n}{adjr2}\PY{p}{,} \PY{n}{xlab}\PY{o}{=}\PY{l+s}{\PYZsq{}}\PY{l+s}{Number of covariates\PYZsq{}}\PY{p}{,}
\PY{n}{ylab}\PY{o}{=}\PY{l+s}{\PYZsq{}}\PY{l+s}{Adjusted R2\PYZsq{}}\PY{p}{,} \PY{n}{type}\PY{o}{=}\PY{l+s}{\PYZsq{}}\PY{l+s}{l\PYZsq{}}\PY{p}{)}
\PY{n}{max.adjr2} \PY{o}{\PYZlt{}\PYZhy{}} \PY{n+nf}{which.max}\PY{p}{(}\PY{n+nf}{summary}\PY{p}{(}\PY{n}{m.forward}\PY{p}{)}\PY{o}{\PYZdl{}}\PY{n}{adjr2}\PY{p}{)}
\PY{n+nf}{points}\PY{p}{(}\PY{n}{max.adjr2}\PY{p}{,} \PY{n+nf}{summary}\PY{p}{(}\PY{n}{m.forward}\PY{p}{)}\PY{o}{\PYZdl{}}\PY{n}{adjr2}\PY{p}{[}\PY{n}{max.adjr2}\PY{p}{]}\PY{p}{,} \PY{n}{col}\PY{o}{=}\PY{l+s}{\PYZsq{}}\PY{l+s}{red\PYZsq{}}\PY{p}{,} \PY{n}{pch}\PY{o}{=}\PY{l+m}{16}\PY{p}{)}
\PY{n+nf}{abline}\PY{p}{(}\PY{n}{v}\PY{o}{=}\PY{n}{max.adjr2}\PY{p}{,} \PY{n}{col}\PY{o}{=}\PY{l+s}{\PYZdq{}}\PY{l+s}{blue\PYZdq{}}\PY{p}{,}\PY{n}{lty}\PY{o}{=}\PY{l+m}{2}\PY{p}{)}


\PY{c+c1}{\PYZsh{}\PYZsh{} BIC}
\PY{n+nf}{plot}\PY{p}{(}\PY{n+nf}{summary}\PY{p}{(}\PY{n}{m.forward}\PY{p}{)}\PY{o}{\PYZdl{}}\PY{n}{bic}\PY{p}{,} \PY{n}{xlab}\PY{o}{=}\PY{l+s}{\PYZsq{}}\PY{l+s}{Number of covariates\PYZsq{}}\PY{p}{,} \PY{n}{ylab}\PY{o}{=}\PY{l+s}{\PYZsq{}}\PY{l+s}{BIC\PYZsq{}}\PY{p}{,} \PY{n}{type}\PY{o}{=}\PY{l+s}{\PYZsq{}}\PY{l+s}{l\PYZsq{}}\PY{p}{)}
\PY{n}{min.bic} \PY{o}{\PYZlt{}\PYZhy{}} \PY{n+nf}{which.min}\PY{p}{(}\PY{n+nf}{summary}\PY{p}{(}\PY{n}{m.forward}\PY{p}{)}\PY{o}{\PYZdl{}}\PY{n}{bic}\PY{p}{)}
\PY{n+nf}{points}\PY{p}{(}\PY{n}{min.bic}\PY{p}{,} \PY{n+nf}{summary}\PY{p}{(}\PY{n}{m.forward}\PY{p}{)}\PY{o}{\PYZdl{}}\PY{n}{bic}\PY{p}{[}\PY{n}{min.bic}\PY{p}{]}\PY{p}{,} \PY{n}{col}\PY{o}{=}\PY{l+s}{\PYZsq{}}\PY{l+s}{red\PYZsq{}}\PY{p}{,} \PY{n}{pch}\PY{o}{=}\PY{l+m}{16}\PY{p}{)}
\PY{n+nf}{abline}\PY{p}{(}\PY{n}{v}\PY{o}{=}\PY{n}{min.bic}\PY{p}{,} \PY{n}{col}\PY{o}{=}\PY{l+s}{\PYZdq{}}\PY{l+s}{blue\PYZdq{}}\PY{p}{,}\PY{n}{lty}\PY{o}{=}\PY{l+m}{2}\PY{p}{)}


\PY{n+nf}{paste0}\PY{p}{(}\PY{l+s}{\PYZdq{}}\PY{l+s}{the number of covariates for the best model base on BIC and AdjR2 are:\PYZdq{}}\PY{p}{,} \PY{l+s}{\PYZdq{}}\PY{l+s}{ \PYZdq{}}\PY{p}{,} \PY{n}{max.adjr2}\PY{p}{,} \PY{l+s}{\PYZdq{}}\PY{l+s}{ \PYZdq{}}\PY{p}{,} \PY{l+s}{\PYZdq{}}\PY{l+s}{for adjr2 and \PYZdq{}}\PY{p}{,} \PY{n}{min.bic}\PY{p}{,} \PY{l+s}{\PYZdq{}}\PY{l+s}{ for BIC\PYZdq{}}\PY{p}{)}
\end{Verbatim}
\end{tcolorbox}

    'the number of covariates for the best model base on BIC and AdjR2 are: 8 for adjr2 and 6 for BIC'

    
    \begin{center}
    \adjustimage{max size={0.9\linewidth}{0.9\paperheight}}{output_43_1.png}
    \end{center}
    { \hspace*{\fill} \\}
    
    base on BIC we keep the model with the lowest BIC so with a number of
variablesequal to : 6.

    \begin{tcolorbox}[breakable, size=fbox, boxrule=1pt, pad at break*=1mm,colback=cellbackground, colframe=cellborder]
\prompt{In}{incolor}{159}{\boxspacing}
\begin{Verbatim}[commandchars=\\\{\}]
\PY{n}{model.bic} \PY{o}{\PYZlt{}\PYZhy{}} \PY{n+nf}{lm}\PY{p}{(}\PY{n}{box} \PY{o}{\PYZti{}} \PY{n}{action}\PY{o}{+}\PY{n}{addict}\PY{o}{+}\PY{n}{cntwait}\PY{o}{+}\PY{n}{budget}\PY{o}{+}\PY{n}{cmngsoon}\PY{o}{+}\PY{n}{sequel}\PY{p}{,} \PY{n}{data}\PY{o}{=}\PY{n}{mydata}\PY{p}{)}
\PY{n+nf}{summary}\PY{p}{(}\PY{n}{model.bic}\PY{p}{)}
\PY{n+nf}{par}\PY{p}{(}\PY{n}{mfrow}\PY{o}{=}\PY{n+nf}{c}\PY{p}{(}\PY{l+m}{2}\PY{p}{,}\PY{l+m}{2}\PY{p}{)}\PY{p}{)}
\PY{n+nf}{plot}\PY{p}{(}\PY{n}{model.bic}\PY{p}{,} \PY{n}{pch}\PY{o}{=}\PY{l+m}{16}\PY{p}{,} \PY{n}{cex}\PY{o}{=}\PY{l+m}{0.7}\PY{p}{,}\PY{n}{col}\PY{o}{=}\PY{l+s}{\PYZdq{}}\PY{l+s}{blue\PYZdq{}}\PY{p}{)}
\end{Verbatim}
\end{tcolorbox}

    
    \begin{Verbatim}[commandchars=\\\{\}]

Call:
lm(formula = box \textasciitilde{} action + addict + cntwait + budget + cmngsoon + 
    sequel, data = mydata)

Residuals:
     Min       1Q   Median       3Q      Max 
-268.797  -56.825   -2.847   65.504  207.729 

Coefficients:
              Estimate Std. Error t value Pr(>|t|)    
(Intercept) -7.599e+01  4.735e+01  -1.605 0.114353    
actionTRUE  -1.280e+02  3.606e+01  -3.550 0.000807 ***
addict       6.403e-03  2.188e-03   2.926 0.005013 ** 
cntwait      3.358e+02  1.212e+02   2.771 0.007644 ** 
budget       1.366e+00  3.854e-01   3.544 0.000822 ***
cmngsoon     3.222e-01  1.587e-01   2.030 0.047288 *  
sequelTRUE   1.016e+02  4.617e+01   2.199 0.032153 *  
---
Signif. codes:  0 ‘***’ 0.001 ‘**’ 0.01 ‘*’ 0.05 ‘.’ 0.1 ‘ ’ 1

Residual standard error: 101.4 on 54 degrees of freedom
Multiple R-squared:  0.696,	Adjusted R-squared:  0.6622 
F-statistic:  20.6 on 6 and 54 DF,  p-value: 2.221e-12

    \end{Verbatim}

    
    \begin{center}
    \adjustimage{max size={0.9\linewidth}{0.9\paperheight}}{output_45_1.png}
    \end{center}
    { \hspace*{\fill} \\}
    
    Now we can judge also our model considering the residuals. The graph of
residuals indicates that the model does not have a good fit. In fact,
the first graph (scatter plot of the residuals) does show a
deterministic pattern . In addition, the mean of the residuals does not
appear to be 0 and the variance of the residuals does not appear to be
constant, as it should be based on the assumptions that the regression
model places on the ε errors. Furthermore, the normality of the
residuals is not satisfied as highlighted in the second graph: the
empirical quantiles in the tails , in fact, do deviate from the
theoretical quantiles of a standard normal. To complete the analysis of
the residuals, no outliers appear to be present: although R highlights
observations, these do not represent outlier observations since Cook's
distance is not large.

    \hypertarget{backword-selection}{%
\subsubsection{BACKWORD SELECTION}\label{backword-selection}}

Let's perform backword selection.

    \begin{tcolorbox}[breakable, size=fbox, boxrule=1pt, pad at break*=1mm,colback=cellbackground, colframe=cellborder]
\prompt{In}{incolor}{151}{\boxspacing}
\begin{Verbatim}[commandchars=\\\{\}]
\PY{n}{m.backward} \PY{o}{\PYZlt{}\PYZhy{}} \PY{n+nf}{regsubsets}\PY{p}{(}\PY{n}{box} \PY{o}{\PYZti{}} \PY{n}{.}\PY{p}{,} \PY{n}{data}\PY{o}{=}\PY{n}{mydata}\PY{p}{,} \PY{n}{nvmax}\PY{o}{=}\PY{l+m}{17}\PY{p}{,} \PY{n}{method}\PY{o}{=}\PY{l+s}{\PYZsq{}}\PY{l+s}{backward\PYZsq{}}\PY{p}{)}
\PY{c+c1}{\PYZsh{}plot(m.backward)}



\PY{n+nf}{par}\PY{p}{(}\PY{n}{mfrow}\PY{o}{=}\PY{n+nf}{c}\PY{p}{(}\PY{l+m}{1}\PY{p}{,}\PY{l+m}{2}\PY{p}{)}\PY{p}{)}
\PY{n+nf}{options}\PY{p}{(}\PY{n}{repr.plot.width} \PY{o}{=} \PY{l+m}{15}\PY{p}{,} \PY{n}{repr.plot.height} \PY{o}{=} \PY{l+m}{10}\PY{p}{)}
\PY{n+nf}{plot}\PY{p}{(}\PY{n}{m.backward}\PY{p}{)}
\PY{n+nf}{plot}\PY{p}{(}\PY{n}{m.backward}\PY{p}{,} \PY{n}{scale}\PY{o}{=}\PY{l+s}{\PYZsq{}}\PY{l+s}{adjr2\PYZsq{}}\PY{p}{)}
\PY{n+nf}{par}\PY{p}{(}\PY{n}{mfrow}\PY{o}{=}\PY{n+nf}{c}\PY{p}{(}\PY{l+m}{2}\PY{p}{,}\PY{l+m}{2}\PY{p}{)}\PY{p}{)}
\PY{c+c1}{\PYZsh{}\PYZsh{} R2}
\PY{n+nf}{plot}\PY{p}{(}\PY{n+nf}{summary}\PY{p}{(}\PY{n}{m.backward}\PY{p}{)}\PY{o}{\PYZdl{}}\PY{n}{rsq}\PY{p}{,} \PY{n}{xlab}\PY{o}{=}\PY{l+s}{\PYZsq{}}\PY{l+s}{Number of covariates\PYZsq{}}\PY{p}{,} \PY{n}{ylab}\PY{o}{=}\PY{l+s}{\PYZsq{}}\PY{l+s}{R2\PYZsq{}}\PY{p}{,} \PY{n}{type}\PY{o}{=}\PY{l+s}{\PYZsq{}}\PY{l+s}{l\PYZsq{}}\PY{p}{)}
\PY{c+c1}{\PYZsh{}\PYZsh{} add on the indication of the best model}
\PY{n}{max.rsq} \PY{o}{\PYZlt{}\PYZhy{}} \PY{n+nf}{which.max}\PY{p}{(}\PY{n+nf}{summary}\PY{p}{(}\PY{n}{m.backward}\PY{p}{)}\PY{o}{\PYZdl{}}\PY{n}{rsq}\PY{p}{)}
\PY{n+nf}{points}\PY{p}{(}\PY{n}{max.rsq}\PY{p}{,} \PY{n+nf}{summary}\PY{p}{(}\PY{n}{m.backward}\PY{p}{)}\PY{o}{\PYZdl{}}\PY{n}{rsq}\PY{p}{[}\PY{n}{max.rsq}\PY{p}{]}\PY{p}{,} \PY{n}{col}\PY{o}{=}\PY{l+s}{\PYZsq{}}\PY{l+s}{red\PYZsq{}}\PY{p}{,} \PY{n}{pch}\PY{o}{=}\PY{l+m}{16}\PY{p}{)}
\PY{n+nf}{abline}\PY{p}{(}\PY{n}{v}\PY{o}{=}\PY{n}{max.rsq}\PY{p}{,} \PY{n}{col}\PY{o}{=}\PY{l+s}{\PYZdq{}}\PY{l+s}{blue\PYZdq{}}\PY{p}{,}\PY{n}{lty}\PY{o}{=}\PY{l+m}{2}\PY{p}{)}


\PY{c+c1}{\PYZsh{}\PYZsh{} RSS}
\PY{n+nf}{plot}\PY{p}{(}\PY{n+nf}{summary}\PY{p}{(}\PY{n}{m.backward}\PY{p}{)}\PY{o}{\PYZdl{}}\PY{n}{rss}\PY{p}{,} \PY{n}{xlab}\PY{o}{=}\PY{l+s}{\PYZsq{}}\PY{l+s}{Number of covariates\PYZsq{}}\PY{p}{,} \PY{n}{ylab}\PY{o}{=}\PY{l+s}{\PYZsq{}}\PY{l+s}{RSS\PYZsq{}}\PY{p}{,} \PY{n}{type}\PY{o}{=}\PY{l+s}{\PYZsq{}}\PY{l+s}{l\PYZsq{}}\PY{p}{)}
\PY{n}{min.rss} \PY{o}{\PYZlt{}\PYZhy{}} \PY{n+nf}{which.min}\PY{p}{(}\PY{n+nf}{summary}\PY{p}{(}\PY{n}{m.backward}\PY{p}{)}\PY{o}{\PYZdl{}}\PY{n}{rss}\PY{p}{)}
\PY{n+nf}{points}\PY{p}{(}\PY{n}{min.rss}\PY{p}{,} \PY{n+nf}{summary}\PY{p}{(}\PY{n}{m.backward}\PY{p}{)}\PY{o}{\PYZdl{}}\PY{n}{rss}\PY{p}{[}\PY{n}{min.rss}\PY{p}{]}\PY{p}{,} \PY{n}{col}\PY{o}{=}\PY{l+s}{\PYZsq{}}\PY{l+s}{red\PYZsq{}}\PY{p}{,} \PY{n}{pch}\PY{o}{=}\PY{l+m}{16}\PY{p}{)}
\PY{n+nf}{abline}\PY{p}{(}\PY{n}{v}\PY{o}{=}\PY{n}{min.rss}\PY{p}{,} \PY{n}{col}\PY{o}{=}\PY{l+s}{\PYZdq{}}\PY{l+s}{blue\PYZdq{}}\PY{p}{,}\PY{n}{lty}\PY{o}{=}\PY{l+m}{2}\PY{p}{)}


\PY{c+c1}{\PYZsh{}\PYZsh{} Adjusted R2}
\PY{n+nf}{plot}\PY{p}{(}\PY{n+nf}{summary}\PY{p}{(}\PY{n}{m.backward}\PY{p}{)}\PY{o}{\PYZdl{}}\PY{n}{adjr2}\PY{p}{,} \PY{n}{xlab}\PY{o}{=}\PY{l+s}{\PYZsq{}}\PY{l+s}{Number of covariates\PYZsq{}}\PY{p}{,}
\PY{n}{ylab}\PY{o}{=}\PY{l+s}{\PYZsq{}}\PY{l+s}{Adjusted R2\PYZsq{}}\PY{p}{,} \PY{n}{type}\PY{o}{=}\PY{l+s}{\PYZsq{}}\PY{l+s}{l\PYZsq{}}\PY{p}{)}
\PY{n}{max.adjr2} \PY{o}{\PYZlt{}\PYZhy{}} \PY{n+nf}{which.max}\PY{p}{(}\PY{n+nf}{summary}\PY{p}{(}\PY{n}{m.backward}\PY{p}{)}\PY{o}{\PYZdl{}}\PY{n}{adjr2}\PY{p}{)}
\PY{n+nf}{points}\PY{p}{(}\PY{n}{max.adjr2}\PY{p}{,} \PY{n+nf}{summary}\PY{p}{(}\PY{n}{m.backward}\PY{p}{)}\PY{o}{\PYZdl{}}\PY{n}{adjr2}\PY{p}{[}\PY{n}{max.adjr2}\PY{p}{]}\PY{p}{,} \PY{n}{col}\PY{o}{=}\PY{l+s}{\PYZsq{}}\PY{l+s}{red\PYZsq{}}\PY{p}{,} \PY{n}{pch}\PY{o}{=}\PY{l+m}{16}\PY{p}{)}
\PY{n+nf}{abline}\PY{p}{(}\PY{n}{v}\PY{o}{=}\PY{n}{max.adjr2}\PY{p}{,} \PY{n}{col}\PY{o}{=}\PY{l+s}{\PYZdq{}}\PY{l+s}{blue\PYZdq{}}\PY{p}{,}\PY{n}{lty}\PY{o}{=}\PY{l+m}{2}\PY{p}{)}


\PY{c+c1}{\PYZsh{}\PYZsh{} BIC}
\PY{n+nf}{plot}\PY{p}{(}\PY{n+nf}{summary}\PY{p}{(}\PY{n}{m.backward}\PY{p}{)}\PY{o}{\PYZdl{}}\PY{n}{bic}\PY{p}{,} \PY{n}{xlab}\PY{o}{=}\PY{l+s}{\PYZsq{}}\PY{l+s}{Number of covariates\PYZsq{}}\PY{p}{,} \PY{n}{ylab}\PY{o}{=}\PY{l+s}{\PYZsq{}}\PY{l+s}{BIC\PYZsq{}}\PY{p}{,} \PY{n}{type}\PY{o}{=}\PY{l+s}{\PYZsq{}}\PY{l+s}{l\PYZsq{}}\PY{p}{)}
\PY{n}{min.bic} \PY{o}{\PYZlt{}\PYZhy{}} \PY{n+nf}{which.min}\PY{p}{(}\PY{n+nf}{summary}\PY{p}{(}\PY{n}{m.backward}\PY{p}{)}\PY{o}{\PYZdl{}}\PY{n}{bic}\PY{p}{)}
\PY{n+nf}{points}\PY{p}{(}\PY{n}{min.bic}\PY{p}{,} \PY{n+nf}{summary}\PY{p}{(}\PY{n}{m.backward}\PY{p}{)}\PY{o}{\PYZdl{}}\PY{n}{bic}\PY{p}{[}\PY{n}{min.bic}\PY{p}{]}\PY{p}{,} \PY{n}{col}\PY{o}{=}\PY{l+s}{\PYZsq{}}\PY{l+s}{red\PYZsq{}}\PY{p}{,} \PY{n}{pch}\PY{o}{=}\PY{l+m}{16}\PY{p}{)}
\PY{n+nf}{abline}\PY{p}{(}\PY{n}{v}\PY{o}{=}\PY{n}{min.bic}\PY{p}{,} \PY{n}{col}\PY{o}{=}\PY{l+s}{\PYZdq{}}\PY{l+s}{blue\PYZdq{}}\PY{p}{,}\PY{n}{lty}\PY{o}{=}\PY{l+m}{2}\PY{p}{)}


\PY{n+nf}{paste0}\PY{p}{(}\PY{l+s}{\PYZdq{}}\PY{l+s}{the number of covariates for the best model base on BIC and AdjR2 are:\PYZdq{}}\PY{p}{,} \PY{l+s}{\PYZdq{}}\PY{l+s}{ \PYZdq{}}\PY{p}{,} \PY{n}{max.adjr2}\PY{p}{,} \PY{l+s}{\PYZdq{}}\PY{l+s}{ \PYZdq{}}\PY{p}{,} \PY{l+s}{\PYZdq{}}\PY{l+s}{for adjr2 and \PYZdq{}}\PY{p}{,} \PY{n}{min.bic}\PY{p}{,} \PY{l+s}{\PYZdq{}}\PY{l+s}{ for BIC\PYZdq{}}\PY{p}{)}
\end{Verbatim}
\end{tcolorbox}

    \begin{center}
    \adjustimage{max size={0.9\linewidth}{0.9\paperheight}}{output_48_0.png}
    \end{center}
    { \hspace*{\fill} \\}
    
    'the number of covariates for the best model base on BIC and AdjR2 are: 8 for adjr2 and 6 for BIC'

    
    \begin{center}
    \adjustimage{max size={0.9\linewidth}{0.9\paperheight}}{output_48_2.png}
    \end{center}
    { \hspace*{\fill} \\}
    
    As you can see, forward and backward give us the same amount of
covariates based on BIC. In fact in this case we have 6 variables.

    \hypertarget{mixed-selection}{%
\subsubsection{MIXED SELECTION}\label{mixed-selection}}

Let's perform mixed selection.

    \begin{tcolorbox}[breakable, size=fbox, boxrule=1pt, pad at break*=1mm,colback=cellbackground, colframe=cellborder]
\prompt{In}{incolor}{153}{\boxspacing}
\begin{Verbatim}[commandchars=\\\{\}]
\PY{n}{m.seqrep} \PY{o}{\PYZlt{}\PYZhy{}} \PY{n+nf}{regsubsets}\PY{p}{(}\PY{n}{box} \PY{o}{\PYZti{}} \PY{n}{.}\PY{p}{,} \PY{n}{data}\PY{o}{=}\PY{n}{mydata}\PY{p}{,} \PY{n}{nvmax}\PY{o}{=}\PY{l+m}{17}\PY{p}{,} \PY{n}{method}\PY{o}{=}\PY{l+s}{\PYZsq{}}\PY{l+s}{seqrep\PYZsq{}}\PY{p}{)}



\PY{n+nf}{par}\PY{p}{(}\PY{n}{mfrow}\PY{o}{=}\PY{n+nf}{c}\PY{p}{(}\PY{l+m}{1}\PY{p}{,}\PY{l+m}{2}\PY{p}{)}\PY{p}{)}
\PY{n+nf}{options}\PY{p}{(}\PY{n}{repr.plot.width} \PY{o}{=} \PY{l+m}{15}\PY{p}{,} \PY{n}{repr.plot.height} \PY{o}{=} \PY{l+m}{10}\PY{p}{)}
\PY{n+nf}{plot}\PY{p}{(}\PY{n}{m.seqrep}\PY{p}{)}
\PY{n+nf}{plot}\PY{p}{(}\PY{n}{m.seqrep}\PY{p}{,} \PY{n}{scale}\PY{o}{=}\PY{l+s}{\PYZsq{}}\PY{l+s}{adjr2\PYZsq{}}\PY{p}{)}

\PY{n+nf}{par}\PY{p}{(}\PY{n}{mfrow}\PY{o}{=}\PY{n+nf}{c}\PY{p}{(}\PY{l+m}{2}\PY{p}{,}\PY{l+m}{2}\PY{p}{)}\PY{p}{)}
\PY{c+c1}{\PYZsh{}\PYZsh{} R2}
\PY{n+nf}{plot}\PY{p}{(}\PY{n+nf}{summary}\PY{p}{(}\PY{n}{m.seqrep}\PY{p}{)}\PY{o}{\PYZdl{}}\PY{n}{rsq}\PY{p}{,} \PY{n}{xlab}\PY{o}{=}\PY{l+s}{\PYZsq{}}\PY{l+s}{Number of covariates\PYZsq{}}\PY{p}{,} \PY{n}{ylab}\PY{o}{=}\PY{l+s}{\PYZsq{}}\PY{l+s}{R2\PYZsq{}}\PY{p}{,} \PY{n}{type}\PY{o}{=}\PY{l+s}{\PYZsq{}}\PY{l+s}{l\PYZsq{}}\PY{p}{)}
\PY{c+c1}{\PYZsh{}\PYZsh{} add on the indication of the best model}
\PY{n}{max.rsq} \PY{o}{\PYZlt{}\PYZhy{}} \PY{n+nf}{which.max}\PY{p}{(}\PY{n+nf}{summary}\PY{p}{(}\PY{n}{m.seqrep}\PY{p}{)}\PY{o}{\PYZdl{}}\PY{n}{rsq}\PY{p}{)}
\PY{n+nf}{points}\PY{p}{(}\PY{n}{max.rsq}\PY{p}{,} \PY{n+nf}{summary}\PY{p}{(}\PY{n}{m.seqrep}\PY{p}{)}\PY{o}{\PYZdl{}}\PY{n}{rsq}\PY{p}{[}\PY{n}{max.rsq}\PY{p}{]}\PY{p}{,} \PY{n}{col}\PY{o}{=}\PY{l+s}{\PYZsq{}}\PY{l+s}{red\PYZsq{}}\PY{p}{,} \PY{n}{pch}\PY{o}{=}\PY{l+m}{16}\PY{p}{)}
\PY{n+nf}{abline}\PY{p}{(}\PY{n}{v}\PY{o}{=}\PY{n}{max.rsq}\PY{p}{,} \PY{n}{col}\PY{o}{=}\PY{l+s}{\PYZdq{}}\PY{l+s}{blue\PYZdq{}}\PY{p}{,}\PY{n}{lty}\PY{o}{=}\PY{l+m}{2}\PY{p}{)}


\PY{c+c1}{\PYZsh{}\PYZsh{} RSS}
\PY{n+nf}{plot}\PY{p}{(}\PY{n+nf}{summary}\PY{p}{(}\PY{n}{m.seqrep}\PY{p}{)}\PY{o}{\PYZdl{}}\PY{n}{rss}\PY{p}{,} \PY{n}{xlab}\PY{o}{=}\PY{l+s}{\PYZsq{}}\PY{l+s}{Number of covariates\PYZsq{}}\PY{p}{,} \PY{n}{ylab}\PY{o}{=}\PY{l+s}{\PYZsq{}}\PY{l+s}{RSS\PYZsq{}}\PY{p}{,} \PY{n}{type}\PY{o}{=}\PY{l+s}{\PYZsq{}}\PY{l+s}{l\PYZsq{}}\PY{p}{)}
\PY{n}{min.rss} \PY{o}{\PYZlt{}\PYZhy{}} \PY{n+nf}{which.min}\PY{p}{(}\PY{n+nf}{summary}\PY{p}{(}\PY{n}{m.seqrep}\PY{p}{)}\PY{o}{\PYZdl{}}\PY{n}{rss}\PY{p}{)}
\PY{n+nf}{points}\PY{p}{(}\PY{n}{min.rss}\PY{p}{,} \PY{n+nf}{summary}\PY{p}{(}\PY{n}{m.seqrep}\PY{p}{)}\PY{o}{\PYZdl{}}\PY{n}{rss}\PY{p}{[}\PY{n}{min.rss}\PY{p}{]}\PY{p}{,} \PY{n}{col}\PY{o}{=}\PY{l+s}{\PYZsq{}}\PY{l+s}{red\PYZsq{}}\PY{p}{,} \PY{n}{pch}\PY{o}{=}\PY{l+m}{16}\PY{p}{)}
\PY{n+nf}{abline}\PY{p}{(}\PY{n}{v}\PY{o}{=}\PY{n}{min.rss}\PY{p}{,} \PY{n}{col}\PY{o}{=}\PY{l+s}{\PYZdq{}}\PY{l+s}{blue\PYZdq{}}\PY{p}{,}\PY{n}{lty}\PY{o}{=}\PY{l+m}{2}\PY{p}{)}


\PY{c+c1}{\PYZsh{}\PYZsh{} Adjusted R2}
\PY{n+nf}{plot}\PY{p}{(}\PY{n+nf}{summary}\PY{p}{(}\PY{n}{m.seqrep}\PY{p}{)}\PY{o}{\PYZdl{}}\PY{n}{adjr2}\PY{p}{,} \PY{n}{xlab}\PY{o}{=}\PY{l+s}{\PYZsq{}}\PY{l+s}{Number of covariates\PYZsq{}}\PY{p}{,}
\PY{n}{ylab}\PY{o}{=}\PY{l+s}{\PYZsq{}}\PY{l+s}{Adjusted R2\PYZsq{}}\PY{p}{,} \PY{n}{type}\PY{o}{=}\PY{l+s}{\PYZsq{}}\PY{l+s}{l\PYZsq{}}\PY{p}{)}
\PY{n}{max.adjr2} \PY{o}{\PYZlt{}\PYZhy{}} \PY{n+nf}{which.max}\PY{p}{(}\PY{n+nf}{summary}\PY{p}{(}\PY{n}{m.seqrep}\PY{p}{)}\PY{o}{\PYZdl{}}\PY{n}{adjr2}\PY{p}{)}
\PY{n+nf}{points}\PY{p}{(}\PY{n}{max.adjr2}\PY{p}{,} \PY{n+nf}{summary}\PY{p}{(}\PY{n}{m.seqrep}\PY{p}{)}\PY{o}{\PYZdl{}}\PY{n}{adjr2}\PY{p}{[}\PY{n}{max.adjr2}\PY{p}{]}\PY{p}{,} \PY{n}{col}\PY{o}{=}\PY{l+s}{\PYZsq{}}\PY{l+s}{red\PYZsq{}}\PY{p}{,} \PY{n}{pch}\PY{o}{=}\PY{l+m}{16}\PY{p}{)}
\PY{n+nf}{abline}\PY{p}{(}\PY{n}{v}\PY{o}{=}\PY{n}{max.adjr2}\PY{p}{,} \PY{n}{col}\PY{o}{=}\PY{l+s}{\PYZdq{}}\PY{l+s}{blue\PYZdq{}}\PY{p}{,}\PY{n}{lty}\PY{o}{=}\PY{l+m}{2}\PY{p}{)}


\PY{c+c1}{\PYZsh{}\PYZsh{} BIC}
\PY{n+nf}{plot}\PY{p}{(}\PY{n+nf}{summary}\PY{p}{(}\PY{n}{m.seqrep}\PY{p}{)}\PY{o}{\PYZdl{}}\PY{n}{bic}\PY{p}{,} \PY{n}{xlab}\PY{o}{=}\PY{l+s}{\PYZsq{}}\PY{l+s}{Number of covariates\PYZsq{}}\PY{p}{,} \PY{n}{ylab}\PY{o}{=}\PY{l+s}{\PYZsq{}}\PY{l+s}{BIC\PYZsq{}}\PY{p}{,} \PY{n}{type}\PY{o}{=}\PY{l+s}{\PYZsq{}}\PY{l+s}{l\PYZsq{}}\PY{p}{)}
\PY{n}{min.bic} \PY{o}{\PYZlt{}\PYZhy{}} \PY{n+nf}{which.min}\PY{p}{(}\PY{n+nf}{summary}\PY{p}{(}\PY{n}{m.seqrep}\PY{p}{)}\PY{o}{\PYZdl{}}\PY{n}{bic}\PY{p}{)}
\PY{n+nf}{points}\PY{p}{(}\PY{n}{min.bic}\PY{p}{,} \PY{n+nf}{summary}\PY{p}{(}\PY{n}{m.seqrep}\PY{p}{)}\PY{o}{\PYZdl{}}\PY{n}{bic}\PY{p}{[}\PY{n}{min.bic}\PY{p}{]}\PY{p}{,} \PY{n}{col}\PY{o}{=}\PY{l+s}{\PYZsq{}}\PY{l+s}{red\PYZsq{}}\PY{p}{,} \PY{n}{pch}\PY{o}{=}\PY{l+m}{16}\PY{p}{)}
\PY{n+nf}{abline}\PY{p}{(}\PY{n}{v}\PY{o}{=}\PY{n}{min.bic}\PY{p}{,} \PY{n}{col}\PY{o}{=}\PY{l+s}{\PYZdq{}}\PY{l+s}{blue\PYZdq{}}\PY{p}{,}\PY{n}{lty}\PY{o}{=}\PY{l+m}{2}\PY{p}{)}


\PY{n+nf}{paste0}\PY{p}{(}\PY{l+s}{\PYZdq{}}\PY{l+s}{the number of covariates for the best model base on BIC and AdjR2 are:\PYZdq{}}\PY{p}{,} \PY{l+s}{\PYZdq{}}\PY{l+s}{ \PYZdq{}}\PY{p}{,} \PY{n}{max.adjr2}\PY{p}{,} \PY{l+s}{\PYZdq{}}\PY{l+s}{ \PYZdq{}}\PY{p}{,} \PY{l+s}{\PYZdq{}}\PY{l+s}{for adjr2 and \PYZdq{}}\PY{p}{,} \PY{n}{min.bic}\PY{p}{,} \PY{l+s}{\PYZdq{}}\PY{l+s}{ for BIC\PYZdq{}}\PY{p}{)}
\end{Verbatim}
\end{tcolorbox}

    \begin{center}
    \adjustimage{max size={0.9\linewidth}{0.9\paperheight}}{output_51_0.png}
    \end{center}
    { \hspace*{\fill} \\}
    
    'the number of covariates for the best model base on BIC and AdjR2 are: 9 for adjr2 and 6 for BIC'

    
    \begin{center}
    \adjustimage{max size={0.9\linewidth}{0.9\paperheight}}{output_51_2.png}
    \end{center}
    { \hspace*{\fill} \\}
    
    As you can see, forward ,backward and mixed selection give us the same
amount of covariates based on BIC

    \hypertarget{principal-component-analysis}{%
\subsubsection{PRINCIPAL COMPONENT
ANALYSIS}\label{principal-component-analysis}}

Let's consider Principal component analysis in order to see if it is
useful. I set the seed at 222.

    \begin{tcolorbox}[breakable, size=fbox, boxrule=1pt, pad at break*=1mm,colback=cellbackground, colframe=cellborder]
\prompt{In}{incolor}{161}{\boxspacing}
\begin{Verbatim}[commandchars=\\\{\}]
\PY{c+c1}{\PYZsh{}\PYZsh{} PCA}
\PY{n+nf}{library}\PY{p}{(}\PY{n}{pls}\PY{p}{)}
\PY{n+nf}{set.seed}\PY{p}{(}\PY{l+m}{222}\PY{p}{)}
\PY{n}{m.pcr} \PY{o}{\PYZlt{}\PYZhy{}} \PY{n+nf}{pcr}\PY{p}{(}\PY{n}{box} \PY{o}{\PYZti{}} \PY{n}{.}\PY{p}{,} \PY{n}{scale}\PY{o}{=}\PY{k+kc}{TRUE}\PY{p}{,} \PY{n}{validation}\PY{o}{=}\PY{l+s}{\PYZsq{}}\PY{l+s}{CV\PYZsq{}}\PY{p}{,} \PY{n}{data}\PY{o}{=}\PY{n}{mydata}\PY{p}{)}
\PY{n+nf}{summary}\PY{p}{(}\PY{n}{m.pcr}\PY{p}{)}
\end{Verbatim}
\end{tcolorbox}

    \begin{Verbatim}[commandchars=\\\{\}]

Attaching package: ‘pls’

The following object is masked from ‘package:stats’:

    loadings

    \end{Verbatim}

    \begin{Verbatim}[commandchars=\\\{\}]
Data:   X dimension: 61 14
        Y dimension: 61 1
Fit method: svdpc
Number of components considered: 14

VALIDATION: RMSEP
Cross-validated using 10 random segments.
       (Intercept)  1 comps  2 comps  3 comps  4 comps  5 comps  6 comps
CV           175.9    132.3    132.4    130.2    135.2    135.2    124.6
adjCV        175.9    131.8    132.0    129.7    134.7    136.0    123.3
       7 comps  8 comps  9 comps  10 comps  11 comps  12 comps  13 comps
CV       123.2    120.9    117.3     119.4     122.9     120.8     122.4
adjCV    122.1    119.9    115.1     118.2     121.6     119.2     120.7
       14 comps
CV        125.2
adjCV     123.3

TRAINING: \% variance explained
     1 comps  2 comps  3 comps  4 comps  5 comps  6 comps  7 comps  8 comps
X      24.46    42.85    55.35    64.45    71.92    78.49    83.74    88.35
box    46.96    47.89    50.58    51.08    51.21    62.12    65.20    66.20
     9 comps  10 comps  11 comps  12 comps  13 comps  14 comps
X      91.61     94.56     96.66     98.42     99.71    100.00
box    69.21     69.34     70.66     71.97     72.06     72.22
    \end{Verbatim}

    The output provides the result of the cross validation in terms of
square root of the MSE for each number of PCs. -Choose the optimum
through a graphical inspection of the results considering MSE and
\(R^{2}\). We see that the number of PCs needed is: 14 While the best
number of components we can use for the analysis based on R comand
SelectNcompo is : 1 We also have that the value of MSE is reported
below.

    \begin{tcolorbox}[breakable, size=fbox, boxrule=1pt, pad at break*=1mm,colback=cellbackground, colframe=cellborder]
\prompt{In}{incolor}{162}{\boxspacing}
\begin{Verbatim}[commandchars=\\\{\}]
\PY{n+nf}{MSEP}\PY{p}{(}\PY{n}{m.pcr}\PY{p}{,} \PY{n}{ncomp}\PY{o}{=}\PY{n+nf}{selectNcomp}\PY{p}{(}\PY{n}{m.pcr}\PY{p}{,} \PY{n}{method}\PY{o}{=}\PY{l+s}{\PYZsq{}}\PY{l+s}{onesigma\PYZsq{}}\PY{p}{,} \PY{n}{ncomp}\PY{o}{=}\PY{l+m}{14}\PY{p}{)}\PY{p}{)}
\end{Verbatim}
\end{tcolorbox}

    
    \begin{Verbatim}[commandchars=\\\{\}]
       (Intercept)  1 comps
CV           30924    17507
adjCV        30924    17376
    \end{Verbatim}

    
    \begin{tcolorbox}[breakable, size=fbox, boxrule=1pt, pad at break*=1mm,colback=cellbackground, colframe=cellborder]
\prompt{In}{incolor}{167}{\boxspacing}
\begin{Verbatim}[commandchars=\\\{\}]
\PY{n+nf}{par}\PY{p}{(}\PY{n}{mfrow}\PY{o}{=}\PY{n+nf}{c}\PY{p}{(}\PY{l+m}{1}\PY{p}{,}\PY{l+m}{2}\PY{p}{)}\PY{p}{)}
\PY{c+c1}{\PYZsh{}\PYZsh{} graph without axes}

\PY{n+nf}{validationplot}\PY{p}{(}\PY{n}{m.pcr}\PY{p}{,} \PY{n}{val.type}\PY{o}{=}\PY{l+s}{\PYZsq{}}\PY{l+s}{MSEP\PYZsq{}}\PY{p}{,} \PY{n}{main}\PY{o}{=}\PY{l+s}{\PYZsq{}}\PY{l+s}{Y\PYZsq{}}\PY{p}{,} \PY{n}{axes}\PY{o}{=}\PY{k+kc}{FALSE}\PY{p}{,} \PY{n}{col}\PY{o}{=}\PY{l+s}{\PYZdq{}}\PY{l+s}{blue\PYZdq{}}\PY{p}{)}
\PY{c+c1}{\PYZsh{}\PYZsh{} add on the x\PYZhy{}axis (1) with the specification (at) of the points at which tick\PYZhy{}mar \PYZsh{}\PYZsh{} are to be drawn}
\PY{n+nf}{axis}\PY{p}{(}\PY{l+m}{1}\PY{p}{,} \PY{n}{at}\PY{o}{=}\PY{l+m}{1}\PY{o}{:}\PY{l+m}{14}\PY{p}{)}
\PY{c+c1}{\PYZsh{}\PYZsh{} add on the y\PYZhy{}axis}
\PY{n+nf}{axis}\PY{p}{(}\PY{l+m}{2}\PY{p}{)}
\PY{n+nf}{text}\PY{p}{(}\PY{n}{x}\PY{o}{=}\PY{l+m}{4}\PY{p}{,} \PY{n}{y}\PY{o}{=}\PY{l+m}{20000}\PY{p}{,}\PY{n}{cex}\PY{o}{=}\PY{l+m}{0.75}\PY{p}{,}\PY{n+nf}{paste0}\PY{p}{(}\PY{l+s}{\PYZdq{}}\PY{l+s}{number of PCs sufficient based on Selectncomp =\PYZdq{}}\PY{p}{,} \PY{n+nf}{selectNcomp}\PY{p}{(}\PY{n}{m.pcr}\PY{p}{,} \PY{n}{method}\PY{o}{=}\PY{l+s}{\PYZsq{}}\PY{l+s}{onesigma\PYZsq{}}\PY{p}{,} \PY{n}{ncomp}\PY{o}{=}\PY{l+m}{14}\PY{p}{)}\PY{p}{)}\PY{p}{)}
\PY{n+nf}{abline}\PY{p}{(}\PY{n}{v}\PY{o}{=}\PY{n+nf}{selectNcomp}\PY{p}{(}\PY{n}{m.pcr}\PY{p}{,} \PY{n}{method}\PY{o}{=}\PY{l+s}{\PYZsq{}}\PY{l+s}{onesigma\PYZsq{}}\PY{p}{,} \PY{n}{ncomp}\PY{o}{=}\PY{l+m}{14}\PY{p}{)}\PY{p}{,} \PY{n}{col}\PY{o}{=}\PY{l+s}{\PYZdq{}}\PY{l+s}{red\PYZdq{}}\PY{p}{,}\PY{n}{lt}\PY{o}{=}\PY{l+m}{2}\PY{p}{)}

\PY{n+nf}{validationplot}\PY{p}{(}\PY{n}{m.pcr}\PY{p}{,} \PY{n}{val.type}\PY{o}{=}\PY{l+s}{\PYZsq{}}\PY{l+s}{R2\PYZsq{}}\PY{p}{,} \PY{n}{main}\PY{o}{=}\PY{l+s}{\PYZsq{}}\PY{l+s}{Y\PYZsq{}}\PY{p}{,} \PY{n}{axes}\PY{o}{=}\PY{k+kc}{FALSE}\PY{p}{,}\PY{n}{col}\PY{o}{=}\PY{l+s}{\PYZdq{}}\PY{l+s}{blue\PYZdq{}}\PY{p}{)}
\PY{n+nf}{axis}\PY{p}{(}\PY{l+m}{1}\PY{p}{,} \PY{n}{at}\PY{o}{=}\PY{l+m}{1}\PY{o}{:}\PY{l+m}{14}\PY{p}{)}
\PY{n+nf}{axis}\PY{p}{(}\PY{l+m}{2}\PY{p}{)}
\PY{n+nf}{text}\PY{p}{(}\PY{n}{x}\PY{o}{=}\PY{l+m}{4}\PY{p}{,} \PY{n}{y}\PY{o}{=}\PY{l+m}{0.3}\PY{p}{,}\PY{n}{cex}\PY{o}{=}\PY{l+m}{0.75}\PY{p}{,}\PY{n+nf}{paste0}\PY{p}{(}\PY{l+s}{\PYZdq{}}\PY{l+s}{number of PCs sufficient based on Selectncomp=\PYZdq{}}\PY{p}{,} \PY{n+nf}{selectNcomp}\PY{p}{(}\PY{n}{m.pcr}\PY{p}{,} \PY{n}{method}\PY{o}{=}\PY{l+s}{\PYZsq{}}\PY{l+s}{onesigma\PYZsq{}}\PY{p}{,} \PY{n}{ncomp}\PY{o}{=}\PY{l+m}{14}\PY{p}{)}\PY{p}{)}\PY{p}{)}
\PY{n+nf}{abline}\PY{p}{(}\PY{n}{v}\PY{o}{=}\PY{n+nf}{selectNcomp}\PY{p}{(}\PY{n}{m.pcr}\PY{p}{,} \PY{n}{method}\PY{o}{=}\PY{l+s}{\PYZsq{}}\PY{l+s}{onesigma\PYZsq{}}\PY{p}{,} \PY{n}{ncomp}\PY{o}{=}\PY{l+m}{14}\PY{p}{)}\PY{p}{,} \PY{n}{col}\PY{o}{=}\PY{l+s}{\PYZdq{}}\PY{l+s}{red\PYZdq{}}\PY{p}{,}\PY{n}{lt}\PY{o}{=}\PY{l+m}{2}\PY{p}{)}
\end{Verbatim}
\end{tcolorbox}

    \begin{center}
    \adjustimage{max size={0.9\linewidth}{0.9\paperheight}}{output_57_0.png}
    \end{center}
    { \hspace*{\fill} \\}
    
    Let's look how much variance is explained by the 14 components in the
plot below. While te explained variance for our 1 PCs obtained before is
25\% that is a bit low. So we will also consider 10 components that are
the same number obtained by lasso which lead to 95\% of explained
deviance.

    \begin{tcolorbox}[breakable, size=fbox, boxrule=1pt, pad at break*=1mm,colback=cellbackground, colframe=cellborder]
\prompt{In}{incolor}{178}{\boxspacing}
\begin{Verbatim}[commandchars=\\\{\}]
\PY{c+c1}{\PYZsh{}\PYZsh{} explained variance}

\PY{n}{sum}\PY{o}{=}\PY{n+nf}{sum}\PY{p}{(}\PY{n+nf}{explvar}\PY{p}{(}\PY{n}{m.pcr}\PY{p}{)}\PY{p}{[}\PY{l+m}{1}\PY{o}{:}\PY{n+nf}{selectNcomp}\PY{p}{(}\PY{n}{m.pcr}\PY{p}{,} \PY{n}{method}\PY{o}{=}\PY{l+s}{\PYZsq{}}\PY{l+s}{onesigma\PYZsq{}}\PY{p}{,} \PY{n}{ncomp}\PY{o}{=}\PY{l+m}{14}\PY{p}{)}\PY{p}{]}\PY{p}{)}

\PY{c+c1}{\PYZsh{}\PYZsh{} plot:}

\PY{n+nf}{plot}\PY{p}{(}\PY{l+m}{1}\PY{o}{:}\PY{l+m}{14}\PY{p}{,} \PY{n+nf}{explvar}\PY{p}{(}\PY{n}{m.pcr}\PY{p}{)}\PY{p}{,} \PY{n}{ylab}\PY{o}{=}\PY{l+s}{\PYZsq{}}\PY{l+s}{Percentage of explained variance\PYZsq{}}\PY{p}{,}
        \PY{n}{xlab}\PY{o}{=}\PY{l+s}{\PYZsq{}}\PY{l+s}{PCs\PYZsq{}}\PY{p}{,} \PY{n}{type}\PY{o}{=}\PY{l+s}{\PYZsq{}}\PY{l+s}{l\PYZsq{}}\PY{p}{,} \PY{n}{axes}\PY{o}{=}\PY{k+kc}{FALSE}\PY{p}{,}\PY{n}{col}\PY{o}{=}\PY{l+s}{\PYZdq{}}\PY{l+s}{blue\PYZdq{}}\PY{p}{,}\PY{n}{main}\PY{o}{=}\PY{l+s}{\PYZdq{}}\PY{l+s}{Explained variance for the PCs\PYZdq{}}\PY{p}{)}
\PY{n+nf}{axis}\PY{p}{(}\PY{l+m}{1}\PY{p}{,} \PY{n}{at}\PY{o}{=}\PY{l+m}{1}\PY{o}{:}\PY{l+m}{14}\PY{p}{)}
\PY{n+nf}{axis}\PY{p}{(}\PY{l+m}{2}\PY{p}{)}
\PY{n+nf}{text}\PY{p}{(}\PY{n}{x}\PY{o}{=}\PY{l+m}{3}\PY{p}{,} \PY{n}{y}\PY{o}{=}\PY{l+m}{5}\PY{p}{,}\PY{n}{cex}\PY{o}{=}\PY{l+m}{0.75}\PY{p}{,}\PY{n+nf}{paste0}\PY{p}{(}\PY{l+s}{\PYZdq{}}\PY{l+s}{explained variance for our 1 PC=\PYZdq{}}\PY{p}{,}\PY{n}{sum}\PY{p}{,}\PY{l+s}{\PYZdq{}}\PY{l+s}{\PYZpc{}\PYZdq{}}\PY{p}{)}\PY{p}{)}
\PY{n+nf}{abline}\PY{p}{(}\PY{n}{v}\PY{o}{=}\PY{n+nf}{selectNcomp}\PY{p}{(}\PY{n}{m.pcr}\PY{p}{,} \PY{n}{method}\PY{o}{=}\PY{l+s}{\PYZsq{}}\PY{l+s}{onesigma\PYZsq{}}\PY{p}{,} \PY{n}{ncomp}\PY{o}{=}\PY{l+m}{14}\PY{p}{)}\PY{p}{,} \PY{n}{col}\PY{o}{=}\PY{l+s}{\PYZdq{}}\PY{l+s}{red\PYZdq{}}\PY{p}{,}\PY{n}{lt}\PY{o}{=}\PY{l+m}{2}\PY{p}{,}\PY{n}{lwd}\PY{o}{=}\PY{l+m}{4}\PY{p}{)}



\PY{n}{x2} \PY{o}{=} \PY{n+nf}{c}\PY{p}{(}\PY{l+m}{0}\PY{p}{,}\PY{l+m}{0}\PY{p}{,}\PY{l+m}{1}\PY{p}{,}\PY{l+m}{1}\PY{p}{)}
\PY{n}{y2} \PY{o}{=} \PY{n+nf}{c}\PY{p}{(}\PY{l+m}{0}\PY{p}{,}\PY{l+m}{100}\PY{p}{,}\PY{n+nf}{explvar}\PY{p}{(}\PY{n}{m.pcr}\PY{p}{)}\PY{p}{[}\PY{l+m}{1}\PY{o}{:}\PY{n+nf}{selectNcomp}\PY{p}{(}\PY{n}{m.pcr}\PY{p}{,} \PY{n}{method}\PY{o}{=}\PY{l+s}{\PYZsq{}}\PY{l+s}{onesigma\PYZsq{}}\PY{p}{,} \PY{n}{ncomp}\PY{o}{=}\PY{l+m}{14}\PY{p}{)}\PY{p}{]}\PY{p}{,}\PY{l+m}{0}\PY{p}{)}
\PY{n+nf}{polygon}\PY{p}{(}\PY{n}{x2}\PY{p}{,}\PY{n}{y2}\PY{p}{,} \PY{n}{col}\PY{o}{=}\PY{l+s}{\PYZdq{}}\PY{l+s}{lightblue\PYZdq{}}\PY{p}{,} \PY{n}{border}\PY{o}{=}\PY{k+kc}{NA}\PY{p}{)}
\end{Verbatim}
\end{tcolorbox}

    \begin{center}
    \adjustimage{max size={0.9\linewidth}{0.9\paperheight}}{output_59_0.png}
    \end{center}
    { \hspace*{\fill} \\}
    
    Let's now plot Plot the regression coefficients associated to the models
with increasing PCs, from 1 to 10 We see that we have our 10 models. We
look for the picks. As picks are higher as our model is better. The
model with 10 comps give us the largest amount of informations (higher
explained deviance).

    \begin{tcolorbox}[breakable, size=fbox, boxrule=1pt, pad at break*=1mm,colback=cellbackground, colframe=cellborder]
\prompt{In}{incolor}{184}{\boxspacing}
\begin{Verbatim}[commandchars=\\\{\}]
\PY{c+c1}{\PYZsh{}plot}

\PY{n+nf}{par}\PY{p}{(}\PY{n}{mfrow}\PY{o}{=}\PY{n+nf}{c}\PY{p}{(}\PY{l+m}{1}\PY{p}{,}\PY{l+m}{2}\PY{p}{)}\PY{p}{)}
\PY{n+nf}{options}\PY{p}{(}\PY{n}{repr.plot.width} \PY{o}{=} \PY{l+m}{15}\PY{p}{,} \PY{n}{repr.plot.height} \PY{o}{=} \PY{l+m}{7}\PY{p}{)}
\PY{n+nf}{coefplot}\PY{p}{(}\PY{n}{m.pcr}\PY{p}{,} \PY{n}{ncomp}\PY{o}{=}\PY{l+m}{1}\PY{o}{:}\PY{l+m}{10}\PY{p}{,} \PY{n}{legendpos}\PY{o}{=}\PY{l+s}{\PYZsq{}}\PY{l+s}{topleft\PYZsq{}}\PY{p}{,} \PY{n}{main}\PY{o}{=}\PY{l+s}{\PYZsq{}}\PY{l+s}{\PYZsq{}}\PY{p}{,}
        \PY{n}{xlab}\PY{o}{=}\PY{l+s}{\PYZsq{}}\PY{l+s}{\PYZsq{}}\PY{p}{,} \PY{n}{ylab}\PY{o}{=}\PY{l+s}{\PYZsq{}}\PY{l+s}{Regression coefficients\PYZsq{}}\PY{p}{,}\PY{n}{xaxt}\PY{o}{=}\PY{l+s}{\PYZdq{}}\PY{l+s}{n\PYZdq{}}\PY{p}{)}
\PY{n+nf}{axis}\PY{p}{(}\PY{l+m}{1}\PY{p}{,} \PY{n}{at}\PY{o}{=}\PY{l+m}{1}\PY{o}{:}\PY{l+m}{12}\PY{p}{,} \PY{n}{labels}\PY{o}{=}\PY{n+nf}{colnames}\PY{p}{(}\PY{n}{mydata}\PY{p}{)}\PY{p}{[}\PY{l+m}{\PYZhy{}1}\PY{p}{]}\PY{p}{,} \PY{n}{las}\PY{o}{=}\PY{l+m}{2}\PY{p}{,} \PY{n}{cex}\PY{o}{=}\PY{l+m}{1}\PY{p}{,} \PY{n}{cex.axis}\PY{o}{=}\PY{l+m}{1}\PY{p}{)}
\PY{n+nf}{title}\PY{p}{(}\PY{l+s}{\PYZdq{}}\PY{l+s}{Regression coefficients associated to the models with increasing PCs\PYZdq{}}\PY{p}{)}

\PY{n+nf}{coefplot}\PY{p}{(}\PY{n}{m.pcr}\PY{p}{,} \PY{n}{ncomp}\PY{o}{=}\PY{l+m}{10}\PY{p}{,} \PY{n}{main}\PY{o}{=}\PY{l+s}{\PYZsq{}}\PY{l+s}{\PYZsq{}}\PY{p}{,} \PY{n}{xlab}\PY{o}{=}\PY{l+s}{\PYZsq{}}\PY{l+s}{\PYZsq{}}\PY{p}{,}\PY{p}{,}\PY{n}{xaxt}\PY{o}{=}\PY{l+s}{\PYZdq{}}\PY{l+s}{n\PYZdq{}}\PY{p}{,} \PY{n}{ylab}\PY{o}{=}\PY{l+s}{\PYZsq{}}\PY{l+s}{Regression coefficients\PYZsq{}}\PY{p}{,}\PY{n}{col}\PY{o}{=}\PY{l+s}{\PYZdq{}}\PY{l+s}{cyan\PYZdq{}}\PY{p}{)}
\PY{n+nf}{title}\PY{p}{(}\PY{l+s}{\PYZdq{}}\PY{l+s}{Regression coefficients associated to 10 Pcs   model\PYZdq{}}\PY{p}{)}
\PY{n+nf}{axis}\PY{p}{(}\PY{l+m}{1}\PY{p}{,} \PY{n}{at}\PY{o}{=}\PY{l+m}{1}\PY{o}{:}\PY{l+m}{12}\PY{p}{,} \PY{n}{labels}\PY{o}{=}\PY{n+nf}{colnames}\PY{p}{(}\PY{n}{mydata}\PY{p}{)}\PY{p}{[}\PY{l+m}{\PYZhy{}1}\PY{p}{]}\PY{p}{,} \PY{n}{las}\PY{o}{=}\PY{l+m}{2}\PY{p}{,} \PY{n}{cex}\PY{o}{=}\PY{l+m}{1}\PY{p}{,} \PY{n}{cex.axis}\PY{o}{=}\PY{l+m}{1}\PY{p}{)}
\end{Verbatim}
\end{tcolorbox}

    \begin{center}
    \adjustimage{max size={0.9\linewidth}{0.9\paperheight}}{output_61_0.png}
    \end{center}
    { \hspace*{\fill} \\}
    
    Let's evaluate the presence of of groups of observations or outliers
through the scores. We do not see a kind of trend in the groups. So PCA
is satisfactory.

    \begin{tcolorbox}[breakable, size=fbox, boxrule=1pt, pad at break*=1mm,colback=cellbackground, colframe=cellborder]
\prompt{In}{incolor}{185}{\boxspacing}
\begin{Verbatim}[commandchars=\\\{\}]
\PY{n+nf}{scoreplot}\PY{p}{(}\PY{n}{m.pcr}\PY{p}{,} \PY{n}{comps}\PY{o}{=}\PY{l+m}{1}\PY{o}{:}\PY{l+m}{5}\PY{p}{,} \PY{n}{cex}\PY{o}{=}\PY{l+m}{0.5}\PY{p}{,} \PY{n}{cex.lab}\PY{o}{=}\PY{l+m}{1.4}\PY{p}{,} \PY{n}{cex.axis}\PY{o}{=}\PY{l+m}{1.4}\PY{p}{,} \PY{n}{pch}\PY{o}{=}\PY{l+m}{19}\PY{p}{,}\PY{n}{col}\PY{o}{=}\PY{l+s}{\PYZdq{}}\PY{l+s}{blue\PYZdq{}}\PY{p}{)}
\PY{n+nf}{scoreplot}\PY{p}{(}\PY{n}{m.pcr}\PY{p}{,} \PY{n}{comps}\PY{o}{=}\PY{l+m}{6}\PY{o}{:}\PY{l+m}{10}\PY{p}{,} \PY{n}{cex}\PY{o}{=}\PY{l+m}{0.5}\PY{p}{,} \PY{n}{cex.lab}\PY{o}{=}\PY{l+m}{1.4}\PY{p}{,} \PY{n}{cex.axis}\PY{o}{=}\PY{l+m}{1.4}\PY{p}{,} \PY{n}{pch}\PY{o}{=}\PY{l+m}{19}\PY{p}{,}\PY{n}{col}\PY{o}{=}\PY{l+s}{\PYZdq{}}\PY{l+s}{blue\PYZdq{}}\PY{p}{)}
\end{Verbatim}
\end{tcolorbox}

    \begin{center}
    \adjustimage{max size={0.9\linewidth}{0.9\paperheight}}{output_63_0.png}
    \end{center}
    { \hspace*{\fill} \\}
    
    \begin{center}
    \adjustimage{max size={0.9\linewidth}{0.9\paperheight}}{output_63_1.png}
    \end{center}
    { \hspace*{\fill} \\}
    
    Finally, evaluate the predictions from the model. Values around the
bisector does suggest a good behavior of the model.

    \begin{tcolorbox}[breakable, size=fbox, boxrule=1pt, pad at break*=1mm,colback=cellbackground, colframe=cellborder]
\prompt{In}{incolor}{186}{\boxspacing}
\begin{Verbatim}[commandchars=\\\{\}]
\PY{c+c1}{\PYZsh{} plot predizione}
\PY{n+nf}{plot}\PY{p}{(}\PY{n}{m.pcr}\PY{p}{,} \PY{n}{xlab}\PY{o}{=}\PY{l+s}{\PYZsq{}}\PY{l+s}{Observed values\PYZsq{}}\PY{p}{,} \PY{n}{ylab}\PY{o}{=}\PY{l+s}{\PYZsq{}}\PY{l+s}{Predictions\PYZsq{}}\PY{p}{,}
        \PY{n}{main}\PY{o}{=}\PY{l+s}{\PYZsq{}}\PY{l+s}{Predictions PCA\PYZsq{}}\PY{p}{,}\PY{n}{col}\PY{o}{=}\PY{l+s}{\PYZdq{}}\PY{l+s}{blue\PYZdq{}}\PY{p}{,}\PY{n}{pch}\PY{o}{=}\PY{l+m}{19}\PY{p}{)}
\PY{n+nf}{abline}\PY{p}{(}\PY{l+m}{0}\PY{p}{,} \PY{l+m}{1}\PY{p}{,} \PY{n}{col}\PY{o}{=}\PY{l+s}{\PYZdq{}}\PY{l+s}{red\PYZdq{}}\PY{p}{,}\PY{n}{lt}\PY{o}{=}\PY{l+m}{2}\PY{p}{,}\PY{n}{lwd}\PY{o}{=}\PY{l+m}{4}\PY{p}{)}
\end{Verbatim}
\end{tcolorbox}

    \begin{center}
    \adjustimage{max size={0.9\linewidth}{0.9\paperheight}}{output_65_0.png}
    \end{center}
    { \hspace*{\fill} \\}
    
    Finally we compute the MSE considering 11 components which is equal to:
9173.

    \hypertarget{conclusion-point-2}{%
\section{Conclusion Point 2}\label{conclusion-point-2}}

Base on MSE we have: - MSE for lasso is:16201

\begin{itemize}
\item
  MSE for linear model is:16332
\item
  MSE for ridge is:16203
\item
  MSE for PCA is: 9173
\end{itemize}

So based on MSE the best approach is with the PCA. In particular we see
that the 10 PC depends on sequel, comedy, horror and fandango. By the
way PCA usually is used for clustering purpose so we might pay attention
about these results. Considering the variable selection lasso usually is
a better approach and in this case it gives a model without the
following covariates:: comedy, animated , fandango, starpower and
mprating3. As concern the automatic selection since the residual of the
selection are not good I decided to not consider it.

    \begin{tcolorbox}[breakable, size=fbox, boxrule=1pt, pad at break*=1mm,colback=cellbackground, colframe=cellborder]
\prompt{In}{incolor}{ }{\boxspacing}
\begin{Verbatim}[commandchars=\\\{\}]

\end{Verbatim}
\end{tcolorbox}


    % Add a bibliography block to the postdoc
    
    
    
\end{document}
